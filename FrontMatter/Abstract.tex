\documentclass[class=report,11pt,crop=false]{standalone}
\input{../Style/ChapterStyle.tex}
\makenoidxglossaries

\newacronym{radar}{RADAR}{Radio Detection and Ranging}
\newacronym{dab}{DAB}{Digital Audio Broadcasting}
\newacronym{fm}{FM}{Frequency Modulation}
\newacronym{am}{AM}{Amplitude Modulation}
\newacronym{fdm}{FDM}{Frequency Division Multiplexing}
\newacronym{ofdm}{OFDM}{Orthogonal Frequency Division Multiplexing}
\newacronym{cofdm}{COFDM}{Coded Orthogonal Frequency Division Multiplexing}
\newacronym{dvbt2}{DVB–T2}{Digital Video Broadcasting — Second Generation Terrestrial}
\newacronym{em}{EM}{electromagnetic}
\newacronym{icasa}{ICASA}{Independent Communications Authority of South Africa}
\newacronym{ioo}{IOO}{Illuminators of Opportunity}
\newacronym{pr}{PR}{Passive Radar}
\newacronym{qpsk}{QPSK}{Quadrature Phase-Shift Keying}
\newacronym{dqpsk}{DQPSK}{Differential~Quadrature~Phase-Shift~Keying}
\newacronym{etsi}{ETSI}{European Telecommunications Standards Institute}
\newacronym{psk}{PSK}{Phase Shift Keying}
\newacronym{ask}{ASK}{Amplitude-Shift Keying}
\newacronym{fsk}{FSK}{Frequency-Shift Keying}
\newacronym{iq}{IQ}{In-phase and Quadrature}
\newacronym{ns}{NS}{Null Symbol}
\newacronym{prs}{PRS}{Phase Reference Symbol}
\newacronym{fic}{FIC}{Fast Information Channel}
\newacronym{msc}{MSC}{Main Service Channel}
\newacronym{dft}{DFT}{Discrete Fourier Transform}
\newacronym{idft}{IDFT}{Inverse Discrete Fourier Transform}
\newacronym{fft}{FFT}{Fast Fourier Transform}
\newacronym{ifft}{IFFT}{Inverse Fast Fourier Transform}
\newacronym{fec}{FEC}{Forward Error Correction}
\newacronym{ard}{ARD}{Amplitude-Range-Doppler}
\newacronym{snr}{SNR}{Signal-to-Noise Ratio}
\newacronym{isi}{ISI}{Intersymbol Interference}
\newacronym{mcm}{MCM}{Multicarrier Modulation}
\begin{document}


% \afterpage{
% \newgeometry{margin=30mm}
\chapter*{Abstract}

\gls{pr} is a variant of radar technology in which no dedicated transmitter is used for operation. Instead, \gls{pr} systems use so-called \gls{ioo}---such as existing radio-station broadcasts---for the detection of objects in a scene. In the past decade, there has been an increased interest in the use of \emph{digital} broadcasting signals as \gls{ioo}s, as these signals offer a variety of benefits to a \gls{pr} system. One of the digital broadcasting standards that has garnered attention for use in \gls{pr} is the \gls{dab} format, which is slowly being adopted around the world to replace the antiquated analogue modes of audio broadcasting.

The aim of this project was to research, design, and implement a \gls{dab} processing chain for the context of a \gls{pr} system. Though prior work has been done on \gls{dab}-based \gls{pr} implementations, there are seemingly no comprehensive descriptions of the salient functions required for the \gls{dab} chain in \gls{pr} systems. Moreover, the literature considering solely \gls{dab} signals focuses too much on the intricacies of the \gls{dab} standard, in a way that is unnecessarily complex for \gls{pr} applications. Therefore, there is a need for a clear and thorough exposition of the design procedure for the \gls{pr}-focused \gls{dab} processing chain.

It begins with a brief surveillance of the available literature, followed by a comprehensive consideration of the relevant aspects of the \gls{dab} standard, including the theoretical concepts that underpin it. Following this is a detailed unpacking of the various components of the \gls{dab} chain, starting from a high-level overview, and then zooming into the functionality of each sub-block in the broader system.

The \gls{dab} chain is then validated via several testing procedures, and is shown to pass all of them. It is thus concluded that the designed processing chain indeed works.

\ifstandalone
\bibliography{../Bibliography/References.bib}
\printnoidxglossary[type=\acronymtype,nonumberlist]
\fi
\end{document}
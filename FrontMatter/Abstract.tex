\documentclass[class=report,11pt,crop=false]{standalone}
% Page geometry
\usepackage[a4paper,margin=25mm,top=25mm,bottom=25mm]{geometry}

% Font choice
\usepackage{lmodern}

% Use IEEE bibliography style
\bibliographystyle{IEEEtran}

% Line spacing
\usepackage{setspace}
\setstretch{1.20}

% Ensure UTF8 encoding
\usepackage[utf8]{inputenc}

% Language standard (not too important)
\usepackage[english]{babel}

% Skip a line in between paragraphs
\usepackage{parskip}

% For the creation of dummy text
\usepackage{blindtext}

% Math
\usepackage{amsmath}

% Header & Footer stuff
\usepackage{fancyhdr}
\pagestyle{fancy}
\fancyhead{}
\fancyhead[R]{\nouppercase{\rightmark}}
\fancyfoot{}
\fancyfoot[C]{\thepage}
\renewcommand{\headrulewidth}{0.0pt}
\renewcommand{\footrulewidth}{0.0pt}
\setlength{\headheight}{13.6pt}

% Page geometry
\usepackage[a4paper,top=25mm,bottom=25mm]{geometry}

% Epigraphs
\usepackage{epigraph}
\setlength\epigraphrule{0pt}

% Hyperlinks & References
\usepackage{hyperref}
\hypersetup{
    colorlinks=true,
    linkcolor=blue,
    filecolor=blue,      
    urlcolor=blue,
    citecolor=blue,
}
\urlstyle{same}

% Automatically correct front-side quotes
\usepackage[autostyle=false, style=american]{csquotes}
\MakeOuterQuote{"}

% Graphics
\usepackage{graphicx}
\graphicspath{{Images/}{../Images/}}

% Colour
\usepackage{color}
\usepackage[usenames,dvipsnames]{xcolor}

% SI units
\usepackage{siunitx}

% Microtype goodness
\usepackage{microtype}

% Listings
\usepackage{listings}
\definecolor{backgroundColour}{RGB}{250,250,250}
\definecolor{commentColour}{RGB}{73, 175, 102}
\definecolor{identifierColour}{RGB}{196, 19, 66}
\definecolor{stringColour}{RGB}{252, 156, 30}
\definecolor{keywordColour}{RGB}{50, 38, 224}
\definecolor{lineNumbersColour}{RGB}{127,127,127}
\lstset{ 
  language=Matlab,
  captionpos=b,
  backgroundcolor=\color{backgroundColour},
  basicstyle=\footnotesize,        % the size of the fonts that are used for the code
  breakatwhitespace=false,         % sets if automatic breaks should only happen at whitespace
  breaklines=true,                 % sets automatic line breaking
  postbreak=\mbox{\textcolor{red}{$\hookrightarrow$}\space},
  commentstyle=\color{commentColour},    % comment style
  identifierstyle=\color{identifierColour},
  stringstyle=\color{stringColour},
   keywordstyle=\color{keywordColour},       % keyword style
  %escapeinside={\%*}{*)},          % if you want to add LaTeX within your code
  extendedchars=true,              % lets you use non-ASCII characters; for 8-bits encodings only, does not work with UTF-8
  frame=single,	                   % adds a frame around the code
  keepspaces=true,                 % keeps spaces in text, useful for keeping indentation of code (possibly needs columns=flexible)
  morekeywords={*,...},            % if you want to add more keywords to the set
  numbers=left,                    % where to put the line-numbers; possible values are (none, left, right)
  numbersep=5pt,                   % how far the line-numbers are from the code
  numberstyle=\tiny\color{lineNumbersColour}, % the style that is used for the line-numbers
  rulecolor=\color{black},         % if not set, the frame-color may be changed on line-breaks within not-black text (e.g. comments (green here))
  showspaces=false,                % show spaces everywhere adding particular underscores; it overrides 'showstringspaces'
  showstringspaces=false,          % underline spaces within strings only
  showtabs=false,                  % show tabs within strings adding particular underscores
  stepnumber=1,                    % the step between two line-numbers. If it's 1, each line will be numbered
  tabsize=2,	                   % sets default tabsize to 2 spaces
  %title=\lstname                   % show the filename of files included with \lstinputlisting; also try caption instead of title
}

% Caption stuff
\usepackage{caption}
\usepackage{subcaption}

\makenoidxglossaries

\newacronym{radar}{RADAR}{Radio Detection and Ranging}
\newacronym{dab}{DAB}{Digital Audio Broadcasting}
\newacronym{fm}{FM}{Frequency Modulation}
\newacronym{am}{AM}{Amplitude Modulation}
\newacronym{fdm}{FDM}{Frequency Division Multiplexing}
\newacronym{ofdm}{OFDM}{Orthogonal Frequency Division Multiplexing}
\newacronym{cofdm}{COFDM}{Coded Orthogonal Frequency Division Multiplexing}
\newacronym{dvbt2}{DVB–T2}{Digital Video Broadcasting — Second Generation Terrestrial}
\newacronym{em}{EM}{electromagnetic}
\newacronym{icasa}{ICASA}{Independent Communications Authority of South Africa}
\newacronym{ioo}{IOO}{Illuminators of Opportunity}
\newacronym{pr}{PR}{Passive Radar}
\newacronym{qpsk}{QPSK}{Differential Quadrature Phase-Shift Keying}
\newacronym{dqpsk}{DQPSK}{Differential Quadrature Phase-Shift Keying}
\newacronym{etsi}{ETSI}{European Telecommunications Standards Institute}
\newacronym{psk}{PSK}{Phase Shift Keying}
\newacronym{ask}{ASK}{Amplitude-Shift Keying}
\newacronym{fsk}{FSK}{Frequency-Shift Keying}
\newacronym{iq}{IQ}{In-phase and Quadrature}
\newacronym{prs}{PRS}{Phase Reference Symbol}
\newacronym{dft}{DFT}{Discrete Fourier Transform}
\newacronym{fft}{FFT}{Fast Fourier Transform}
\begin{document}


% \afterpage{
% \newgeometry{margin=30mm}
\chapter*{Abstract}

\gls{pr} is a variant of radar technology in which no dedicated transmitter is required for operation. Instead, \gls{pr} systems use so-called \gls{ioo}---such as existing radio-station broadcasts---for the detection of objects in a scene. In the past decade, there has been an increased interest in the use of \emph{digital} broadcasting signals as \gls{ioo}s, as these signals offer a variety of benefits in a \gls{pr} context. One of the digital broadcasting standards that has garnered attention for use in \gls{pr} is the \gls{dab} format, which is slowly being adopted around the world to replace the antiquated analogue-modes of audio-broadcasting.

The aim of this project was to research, design, and implement a \gls{dab} processing chain, with the intention that it eventually be integrated into a \gls{pr} system. Though prior work has been done on \gls{dab}-based \gls{pr} implementations, the literature lacks a clear description of the \gls{dab} pipeline used for such scenarios. Moreover, the work that covers \gls{dab} signals more comprehensively focuses too much on the intricacies of the \gls{dab} standard, in a way that is unnecessarily complex for \gls{pr} applications. Therefore, there is a need for a clear and thorough exposition of the design procedure for a \gls{dab} processing chain, specifically within the context of \gls{pr}.

This report aims to document the project's design steps, and its associated results. After a brief surveillance of the available literature, a comprehensive consideration of the relevant aspects of the \gls{dab} standard is provided, including the key theoretical concepts that underpin it---namely, \gls{cofdm} and \gls{dqpsk}. Thereafter, a detailed unpacking of the designed \gls{dab} processing chain is provided, starting from a high-level overview, and then zooming in to the individual functional blocks. The \gls{dab} chain is then validated via several testing procedures, and is shown to pass all of them. It is thus concluded that the designed processing chain indeed works.

\ifstandalone
% \bibliography{../Bibliography/References.bib}
% \printnoidxglossary[type=\acronymtype,nonumberlist]
\fi
\end{document}
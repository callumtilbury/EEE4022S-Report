\chapter*{Acknowledgements}

But how can we find ourselves again? How can man know himself? He is a dark and veiled thing; and if the hare has seven skins, man can shed seventy times seven and still not be able to say: "this is really you, this is no longer slough." In addition, it is a painful and dangerous mission to tunnel into oneself and make a forced descent into the shaft of one's being by the nearest path. Doing so can easily cause damage that no physician can heal. And besides: what need should there be for it, when given all the evidence of our nature, our friendships and enmities, our glance and the clasp of our hand, our memory and that which we forget, our books and our handwriting. This, however, is the means to plan the most important inquiry. Let the youthful soul look back on life with the question: what have you truly loved up to now, what has elevated your soul, what has mastered it and at the same time delighted it? Place these venerated objects before you in a row, and perhaps they will yield for you, through their nature and their sequence, a law, the fundamental law of your true self. Compare these objects, see how one complements, expands, surpasses, transfigures another, how they form a stepladder upon which you have climbed up to yourself as you are now; for your true nature lies, not hidden deep within you, but immeasurably high above you, or at least above that which you normally take to be yourself. Your true educators and formative teachers reveal to you what the real raw material of your being is, something quite ineducable, yet in any case accessible only with difficulty, bound, paralyzed: your educators can be only your liberators. And that is the secret of all education: it does not lend artificial limbs, wax noses or spectacled eyes--rather, what can give these things is only the afterimage of education. But liberation is: the clearing away of all weeds, debris, vermin--that want to infringe upon the tender buds of the plant--an effusion of light and warmth, the gentle, quiet rustling of nocturnal rain, it is imitation and worship of nature, where nature is disposed to being motherly and merciful, it is the perfecting of nature when it prevents her cruel and merciless attacks and turns them to good, when it draws a veil over the expressions of nature's stepmotherly disposition and her sad lack of understanding. — Friedrich Nietzsche
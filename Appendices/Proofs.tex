% ----------------------------------------------------
% Proofs
% ----------------------------------------------------
\documentclass[class=report,11pt,crop=false]{standalone}
% Page geometry
\usepackage[a4paper,margin=25mm,top=25mm,bottom=25mm]{geometry}

% Font choice
\usepackage{lmodern}

% Use IEEE bibliography style
\bibliographystyle{IEEEtran}

% Line spacing
\usepackage{setspace}
\setstretch{1.20}

% Ensure UTF8 encoding
\usepackage[utf8]{inputenc}

% Language standard (not too important)
\usepackage[english]{babel}

% Skip a line in between paragraphs
\usepackage{parskip}

% For the creation of dummy text
\usepackage{blindtext}

% Math
\usepackage{amsmath}

% Header & Footer stuff
\usepackage{fancyhdr}
\pagestyle{fancy}
\fancyhead{}
\fancyhead[R]{\nouppercase{\rightmark}}
\fancyfoot{}
\fancyfoot[C]{\thepage}
\renewcommand{\headrulewidth}{0.0pt}
\renewcommand{\footrulewidth}{0.0pt}
\setlength{\headheight}{13.6pt}

% Page geometry
\usepackage[a4paper,top=25mm,bottom=25mm]{geometry}

% Epigraphs
\usepackage{epigraph}
\setlength\epigraphrule{0pt}

% Hyperlinks & References
\usepackage{hyperref}
\hypersetup{
    colorlinks=true,
    linkcolor=blue,
    filecolor=blue,      
    urlcolor=blue,
    citecolor=blue,
}
\urlstyle{same}

% Automatically correct front-side quotes
\usepackage[autostyle=false, style=american]{csquotes}
\MakeOuterQuote{"}

% Graphics
\usepackage{graphicx}
\graphicspath{{Images/}{../Images/}}

% Colour
\usepackage{color}
\usepackage[usenames,dvipsnames]{xcolor}

% SI units
\usepackage{siunitx}

% Microtype goodness
\usepackage{microtype}

% Listings
\usepackage{listings}
\definecolor{backgroundColour}{RGB}{250,250,250}
\definecolor{commentColour}{RGB}{73, 175, 102}
\definecolor{identifierColour}{RGB}{196, 19, 66}
\definecolor{stringColour}{RGB}{252, 156, 30}
\definecolor{keywordColour}{RGB}{50, 38, 224}
\definecolor{lineNumbersColour}{RGB}{127,127,127}
\lstset{ 
  language=Matlab,
  captionpos=b,
  backgroundcolor=\color{backgroundColour},
  basicstyle=\footnotesize,        % the size of the fonts that are used for the code
  breakatwhitespace=false,         % sets if automatic breaks should only happen at whitespace
  breaklines=true,                 % sets automatic line breaking
  postbreak=\mbox{\textcolor{red}{$\hookrightarrow$}\space},
  commentstyle=\color{commentColour},    % comment style
  identifierstyle=\color{identifierColour},
  stringstyle=\color{stringColour},
   keywordstyle=\color{keywordColour},       % keyword style
  %escapeinside={\%*}{*)},          % if you want to add LaTeX within your code
  extendedchars=true,              % lets you use non-ASCII characters; for 8-bits encodings only, does not work with UTF-8
  frame=single,	                   % adds a frame around the code
  keepspaces=true,                 % keeps spaces in text, useful for keeping indentation of code (possibly needs columns=flexible)
  morekeywords={*,...},            % if you want to add more keywords to the set
  numbers=left,                    % where to put the line-numbers; possible values are (none, left, right)
  numbersep=5pt,                   % how far the line-numbers are from the code
  numberstyle=\tiny\color{lineNumbersColour}, % the style that is used for the line-numbers
  rulecolor=\color{black},         % if not set, the frame-color may be changed on line-breaks within not-black text (e.g. comments (green here))
  showspaces=false,                % show spaces everywhere adding particular underscores; it overrides 'showstringspaces'
  showstringspaces=false,          % underline spaces within strings only
  showtabs=false,                  % show tabs within strings adding particular underscores
  stepnumber=1,                    % the step between two line-numbers. If it's 1, each line will be numbered
  tabsize=2,	                   % sets default tabsize to 2 spaces
  %title=\lstname                   % show the filename of files included with \lstinputlisting; also try caption instead of title
}

% Caption stuff
\usepackage{caption}
\usepackage{subcaption}

\makenoidxglossaries

\newacronym{radar}{RADAR}{Radio Detection and Ranging}
\newacronym{dab}{DAB}{Digital Audio Broadcasting}
\newacronym{fm}{FM}{Frequency Modulation}
\newacronym{am}{AM}{Amplitude Modulation}
\newacronym{fdm}{FDM}{Frequency Division Multiplexing}
\newacronym{ofdm}{OFDM}{Orthogonal Frequency Division Multiplexing}
\newacronym{cofdm}{COFDM}{Coded Orthogonal Frequency Division Multiplexing}
\newacronym{dvbt2}{DVB–T2}{Digital Video Broadcasting — Second Generation Terrestrial}
\newacronym{em}{EM}{electromagnetic}
\newacronym{icasa}{ICASA}{Independent Communications Authority of South Africa}
\newacronym{ioo}{IOO}{Illuminators of Opportunity}
\newacronym{pr}{PR}{Passive Radar}
\newacronym{qpsk}{QPSK}{Differential Quadrature Phase-Shift Keying}
\newacronym{dqpsk}{DQPSK}{Differential Quadrature Phase-Shift Keying}
\newacronym{etsi}{ETSI}{European Telecommunications Standards Institute}
\newacronym{psk}{PSK}{Phase Shift Keying}
\newacronym{ask}{ASK}{Amplitude-Shift Keying}
\newacronym{fsk}{FSK}{Frequency-Shift Keying}
\newacronym{iq}{IQ}{In-phase and Quadrature}
\newacronym{prs}{PRS}{Phase Reference Symbol}
\newacronym{dft}{DFT}{Discrete Fourier Transform}
\newacronym{fft}{FFT}{Fast Fourier Transform}
\begin{document}
% ----------------------------------------------------
\chapter{Proof of \glsentrytext{ofdm} Carrier Orthogonality \label{sect:proofs_ofdm-orthog}}
% ----------------------------------------------------

\emph{Statement:}
Given two complex exponential signals with frequencies that differ by \(\Delta\omega = \frac{2\pi k}{T_u}\), the signals are orthogonal over an integration period of~\(T_u\) for any integer value of~\(k\).

\emph{Proof:}
% \begin{proof}
    Consider the two complex exponential signals that differ by a frequency of \(\Delta\omega\),
    \begin{align}
        \psi_1 &= \rho_1 \cdot e^{j(\omega_0 t + \theta_1)} \\
        \psi_2 &= \rho_2 \cdot e^{j((\omega_0 + \Delta\omega)t + \theta_2)}
    \end{align}
    where \(\rho_1\) and \(\rho_2\) are arbitrary real values, and \(\theta_1\) and  \(\theta_2\) are arbitrary phase-shifts.

    The inner product of the two signals over an integration period of \(T_u\) is defined as,
    % \begin{equation}
        \setlength{\jot}{10pt}
        \begin{align}
            \langle \psi_1, \psi_2 \rangle &:= \int_0^{T_u} \psi_1(t) \psi_2^*(t) \: dt
        \end{align}
    where \(\psi_2^*(t)\) is the complex conjugate of \(\psi_2(t)\). Substituting \(\psi_1\) and \(\psi_2\), and then simplifying,
    \begin{align}
        \langle \psi_1, \psi_2 \rangle &= \int_0^{T_u} \rho_1 e^{j(\omega_0 t + \theta_1)} \cdot \rho_2 e^{-j((\omega_0 + \Delta\omega)t + \theta_2)} \: dt \\
        &= \rho_1 \rho_2 \int_0^{T_u} e^{j(\omega_0 t + \theta_1 - \omega_0 t - \Delta\omega t - \theta_2)} \: dt \\
        &= \rho_1 \rho_2 e^{j(\theta_1 - \theta_2)} \int_0^{T_u} e^{-j \Delta\omega t} \: dt
    \end{align}

    Letting \(C = \rho_1 \rho_2 e^{j(\theta_1 - \theta_2)}\), substituting \(\Delta\omega = \frac{2\pi k}{T_u}\), and calculating the definite integral,
    \begin{align}
        \langle \psi_1, \psi_2 \rangle &= C \cdot \frac{1}{-j2\pi\frac{k}{T_u}} \bigg[ e^{-j2\pi\frac{k}{T_u} t} \bigg]^{t=T_u}_{t=0} \\
        &= C \cdot \frac{1}{-j2\pi\frac{k}{T_u}} (e^{-j2\pi k} - 1)
    \end{align}
    Since \(e^{-j2\pi k} = 1 \: \forall \: k \in \mathbb{Z}\),
    \begin{align}
        \langle \psi_1, \psi_2 \rangle &= 0
    \end{align}
    Therefore, \(\psi_1\) and \(\psi_2\) are orthogonal over the integration period \(T_u\).
% \end{proof}

% ----------------------------------------------------
\ifstandalone
% \bibliography{../Bibliography/References.bib}
\printnoidxglossary[type=\acronymtype,nonumberlist]
\fi
\end{document}
% ----------------------------------------------------
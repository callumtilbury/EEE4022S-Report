% ----------------------------------------------------
% Proofs
% ----------------------------------------------------
\documentclass[class=report,11pt,crop=false]{standalone}
\input{../Style/ChapterStyle.tex}
\makenoidxglossaries

\newacronym{radar}{RADAR}{Radio Detection and Ranging}
\newacronym{dab}{DAB}{Digital Audio Broadcasting}
\newacronym{fm}{FM}{Frequency Modulation}
\newacronym{am}{AM}{Amplitude Modulation}
\newacronym{fdm}{FDM}{Frequency Division Multiplexing}
\newacronym{ofdm}{OFDM}{Orthogonal Frequency Division Multiplexing}
\newacronym{cofdm}{COFDM}{Coded Orthogonal Frequency Division Multiplexing}
\newacronym{dvbt2}{DVB–T2}{Digital Video Broadcasting — Second Generation Terrestrial}
\newacronym{em}{EM}{electromagnetic}
\newacronym{icasa}{ICASA}{Independent Communications Authority of South Africa}
\newacronym{ioo}{IOO}{Illuminators of Opportunity}
\newacronym{pr}{PR}{Passive Radar}
\newacronym{qpsk}{QPSK}{Quadrature Phase-Shift Keying}
\newacronym{dqpsk}{DQPSK}{Differential~Quadrature~Phase-Shift~Keying}
\newacronym{etsi}{ETSI}{European Telecommunications Standards Institute}
\newacronym{psk}{PSK}{Phase Shift Keying}
\newacronym{ask}{ASK}{Amplitude-Shift Keying}
\newacronym{fsk}{FSK}{Frequency-Shift Keying}
\newacronym{iq}{IQ}{In-phase and Quadrature}
\newacronym{ns}{NS}{Null Symbol}
\newacronym{prs}{PRS}{Phase Reference Symbol}
\newacronym{fic}{FIC}{Fast Information Channel}
\newacronym{msc}{MSC}{Main Service Channel}
\newacronym{dft}{DFT}{Discrete Fourier Transform}
\newacronym{idft}{IDFT}{Inverse Discrete Fourier Transform}
\newacronym{fft}{FFT}{Fast Fourier Transform}
\newacronym{ifft}{IFFT}{Inverse Fast Fourier Transform}
\newacronym{fec}{FEC}{Forward Error Correction}
\newacronym{ard}{ARD}{Amplitude-Range-Doppler}
\newacronym{snr}{SNR}{Signal-to-Noise Ratio}
\newacronym{isi}{ISI}{Intersymbol Interference}
\newacronym{mcm}{MCM}{Multicarrier Modulation}
\begin{document}
% ----------------------------------------------------
\chapter{Proof of \glsentrytext{ofdm} Carrier Orthogonality \label{sect:proofs_ofdm-orthog}}
% ----------------------------------------------------

\emph{Statement:}
Given two complex exponential signals with frequencies that differ by \(\Delta\omega = \frac{2\pi k}{T_u}\), the signals are orthogonal over an integration period of~\(T_u\) for any integer value of~\(k\).

\emph{Proof:}
% \begin{proof}
    Consider the two complex exponential signals that differ by a frequency of \(\Delta\omega\),
    \begin{align}
        \psi_1 &= \rho_1 \cdot e^{j(\omega_0 t + \theta_1)} \\
        \psi_2 &= \rho_2 \cdot e^{j((\omega_0 + \Delta\omega)t + \theta_2)}
    \end{align}
    where \(\rho_1\) and \(\rho_2\) are arbitrary real values, and \(\theta_1\) and  \(\theta_2\) are arbitrary phase-shifts.

    The inner product of the two signals over an integration period of \(T_u\) is defined as,
    % \begin{equation}
        \setlength{\jot}{10pt}
        \begin{align}
            \langle \psi_1, \psi_2 \rangle &:= \int_0^{T_u} \psi_1(t) \psi_2^*(t) \: dt
        \end{align}
    where \(\psi_2^*(t)\) is the complex conjugate of \(\psi_2(t)\). Substituting \(\psi_1\) and \(\psi_2\), and then simplifying,
    \begin{align}
        \langle \psi_1, \psi_2 \rangle &= \int_0^{T_u} \rho_1 e^{j(\omega_0 t + \theta_1)} \cdot \rho_2 e^{-j((\omega_0 + \Delta\omega)t + \theta_2)} \: dt \\
        &= \rho_1 \rho_2 \int_0^{T_u} e^{j(\omega_0 t + \theta_1 - \omega_0 t - \Delta\omega t - \theta_2)} \: dt \\
        &= \rho_1 \rho_2 e^{j(\theta_1 - \theta_2)} \int_0^{T_u} e^{-j \Delta\omega t} \: dt
    \end{align}

    Letting \(C = \rho_1 \rho_2 e^{j(\theta_1 - \theta_2)}\), substituting \(\Delta\omega = \frac{2\pi k}{T_u}\), and calculating the definite integral,
    \begin{align}
        \langle \psi_1, \psi_2 \rangle &= C \cdot \frac{1}{-j2\pi\frac{k}{T_u}} \bigg[ e^{-j2\pi\frac{k}{T_u} t} \bigg]^{t=T_u}_{t=0} \\
        &= C \cdot \frac{1}{-j2\pi\frac{k}{T_u}} (e^{-j2\pi k} - 1)
    \end{align}
    Since \(e^{-j2\pi k} = 1 \: \forall \: k \in \mathbb{Z}\),
    \begin{align}
        \langle \psi_1, \psi_2 \rangle &= 0
    \end{align}
    Therefore, \(\psi_1\) and \(\psi_2\) are orthogonal over the integration period \(T_u\).
% \end{proof}

% ----------------------------------------------------
\ifstandalone
% \bibliography{../Bibliography/References.bib}
\printnoidxglossary[type=\acronymtype,nonumberlist]
\fi
\end{document}
% ----------------------------------------------------
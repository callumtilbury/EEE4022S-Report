% ----------------------------------------------------
% Introduction
% ----------------------------------------------------
\documentclass[class=report,11pt,crop=false]{standalone}
\input{../Style/ChapterStyle.tex}
\makenoidxglossaries

\newacronym{radar}{RADAR}{Radio Detection and Ranging}
\newacronym{dab}{DAB}{Digital Audio Broadcasting}
\newacronym{fm}{FM}{Frequency Modulation}
\newacronym{am}{AM}{Amplitude Modulation}
\newacronym{fdm}{FDM}{Frequency Division Multiplexing}
\newacronym{ofdm}{OFDM}{Orthogonal Frequency Division Multiplexing}
\newacronym{cofdm}{COFDM}{Coded Orthogonal Frequency Division Multiplexing}
\newacronym{dvbt2}{DVB–T2}{Digital Video Broadcasting — Second Generation Terrestrial}
\newacronym{em}{EM}{electromagnetic}
\newacronym{icasa}{ICASA}{Independent Communications Authority of South Africa}
\newacronym{ioo}{IOO}{Illuminators of Opportunity}
\newacronym{pr}{PR}{Passive Radar}
\newacronym{qpsk}{QPSK}{Quadrature Phase-Shift Keying}
\newacronym{dqpsk}{DQPSK}{Differential~Quadrature~Phase-Shift~Keying}
\newacronym{etsi}{ETSI}{European Telecommunications Standards Institute}
\newacronym{psk}{PSK}{Phase Shift Keying}
\newacronym{ask}{ASK}{Amplitude-Shift Keying}
\newacronym{fsk}{FSK}{Frequency-Shift Keying}
\newacronym{iq}{IQ}{In-phase and Quadrature}
\newacronym{ns}{NS}{Null Symbol}
\newacronym{prs}{PRS}{Phase Reference Symbol}
\newacronym{fic}{FIC}{Fast Information Channel}
\newacronym{msc}{MSC}{Main Service Channel}
\newacronym{dft}{DFT}{Discrete Fourier Transform}
\newacronym{idft}{IDFT}{Inverse Discrete Fourier Transform}
\newacronym{fft}{FFT}{Fast Fourier Transform}
\newacronym{ifft}{IFFT}{Inverse Fast Fourier Transform}
\newacronym{fec}{FEC}{Forward Error Correction}
\newacronym{ard}{ARD}{Amplitude-Range-Doppler}
\newacronym{snr}{SNR}{Signal-to-Noise Ratio}
\newacronym{isi}{ISI}{Intersymbol Interference}
\newacronym{mcm}{MCM}{Multicarrier Modulation}
\begin{document}
% ----------------------------------------------------
\chapter{Introduction}
\epigraph{Philosophers have hitherto only interpreted the world in various ways; the point is to change it.}%
    {\emph{---Karl Marx}}
% ----------------------------------------------------

This report describes the research, design, and validation steps taken to implement a processing chain for \gls{dab} signals, specifically oriented for the context of \gls{pr}. This did not entail a creating \emph{complete} \gls{dab} pipeline---from a received signal to extracted audio---but rather, focused on the demodulation and subsequent remodulation of \gls{dab} data. The intention behind such a process was to generate "perfect" reference data from received surveillance data in a \gls{pr} system.

\section{Background}
Fundamentally, radar\footnote{The word \emph{radar} was originally derived from the acronym, \acrlong{radar}; however, it has subsequently become a common noun, and is used in this report as such.} is a fairly straightforward concept, one that is analogous to the biological mechanism of \emph{echolocation}---used by animals such as bats and dolphins. In the simplest form, a radar system consists of a co-located transmitter and receiver. If the transmitter sends out an \gls{em} signal---a short pulse, for example---this signal will propagate through space, hitting objects as it does. The collection of objects is often called a \emph{scene}. Depending on the properties of the objects the \gls{em} waves hit, some of the energy will reflect off them, thus travelling back to the receiver. There will be a variety of time delays between sending out the signal and receiving the reflections, depending on the positions of the objects. Since these waves are travelling at the speed of light, which is known,\footnote{The speed of light in a vacuum, \(c\), is known exactly, as our measurement of distance is actually defined in terms of \(c\), not the other way around. The speed of light in air, however, is slightly slower than this---and is thus an approximation. Nevertheless, this difference is usually negligible.} the time delays can be converted to distances---thus revealing the ranges to the various objects. Additional complexity can be introduced to calculate the objects' actual positions, velocities, and so on. This is the basis for an \emph{active} radar system---because the transmitter is \emph{actively} producing the \gls{em} waves.

A \gls{pr} system, on the other hand, is different: instead of having a transmitter and receiver, it only has the latter. Rather than actively transmitting signals, it uses existing \gls{em} transmissions---sometimes called \gls{ioo}---that are being transmitted by some other operator---a radio station, for example. While the mathematics is slightly more complicated for this approach, the core idea remains the same: reflections of the transmitted signal are measured, and using the calculated time delays, the locations of objects are calculated. This approach is sometimes called Passive Coherent Location, Passive Bistatic Radar, or Commensal Radar. The simple term, \gls{pr}, will be used throughout this report.

For the past century of radar development, much of the focus has been on \emph{active} radar systems, despite \gls{pr} systems appearing as early as 1935. However, in recent decades, the use of \gls{pr} has been shown to have a host of advantages over its active counterpart. As a result, there has been a renewed interest in the field, across a broad range of use cases.

A substantial amount of the research done in \gls{pr} has involved the use of analogue broadcasting signals as \gls{ioo}s, such as \gls{fm} signals---likely because they are so ubiquitous worldwide. Nonetheless, these transmissions have a variety of downsides when implemented in a \gls{pr} system, such as inconsistent range-resolution in the case of \gls{fm}. These downsides have motivated the investigation into the use of \emph{digital} broadcasting signals as \gls{ioo}s instead. Digital signals help mitigate some of the challenges encountered when using analogue signals, and can provide additional benefits to \gls{pr} systems.

One such digital broadcasting standard, \gls{dab}, has been shown to yield promising results in \gls{pr} scenarios. However, though work has been published involving the use of \gls{dab} signals in \gls{pr} situations, there lacks a comprehensive guide for doing so. The design of the \gls{dab} processing chain is usually glossed over in favour of system results. Therefore, there is a need for a clear and coherent description of using \gls{dab} signals in these contexts, via a thorough exposition of the relevant processing chain design.

\section{Objectives}
The objectives of this project were: firstly, to gain an understanding

to design and build the demodulation and remodulation components of a \gls{dab} processing chain. Moreover, these 

\section{System Requirements}

\section{Scope \& Limitations}
% The project for which this report is written was considerably time-bound---with approximately twelve weeks from start to finish. Owing to this, the study was intentionally kept fairly narrow in scope. Though the field of passive radar formed an important underpinning for the intentions of the project, actual passive radar \emph{implementation} was not a primary focus. Instead, with the radar requirements in mind, the project focused on manipulating \acrshort{dab} signals in a useful way.

\subsection{Scope}
Formally, the scope of this project included:
\begin{itemize}
    \item 
\end{itemize}

\subsection{Limitations}
% In any given year, there are a host of major limitations to undergraduate projects. This year, these limitations were massively aggravated by the ongoing global COVID-19 pandemic. They include:
% \begin{itemize}
%     \item Lack of lab equipment access
%     \item In-person consultation difficulties
%     \item Budget constraints
% \end{itemize}


\section{Report Outline}
Chapter~\ref{ch:literature}

Chapter~\ref{ch:dab-standard}

Chapter~\ref{ch:dab-processing}

Chapter~\ref{ch:dab-validation}

Chapter~\ref{ch:pr-integration}

Chapter~\ref{ch:conclusions}

Chapter~\ref{ch:recommendations}



% ----------------------------------------------------
\ifstandalone
\bibliography{../Bibliography/References.bib}
\printnoidxglossary[type=\acronymtype,nonumberlist]
\fi
\end{document}
% ----------------------------------------------------
% ----------------------------------------------------
% Introduction
% ----------------------------------------------------
\documentclass[class=report,11pt,crop=false]{standalone}
% Page geometry
\usepackage[a4paper,margin=25mm,top=25mm,bottom=25mm]{geometry}

% Font choice
\usepackage{lmodern}

% Use IEEE bibliography style
\bibliographystyle{IEEEtran}

% Line spacing
\usepackage{setspace}
\setstretch{1.20}

% Ensure UTF8 encoding
\usepackage[utf8]{inputenc}

% Language standard (not too important)
\usepackage[english]{babel}

% Skip a line in between paragraphs
\usepackage{parskip}

% For the creation of dummy text
\usepackage{blindtext}

% Math
\usepackage{amsmath}

% Header & Footer stuff
\usepackage{fancyhdr}
\pagestyle{fancy}
\fancyhead{}
\fancyhead[R]{\nouppercase{\rightmark}}
\fancyfoot{}
\fancyfoot[C]{\thepage}
\renewcommand{\headrulewidth}{0.0pt}
\renewcommand{\footrulewidth}{0.0pt}
\setlength{\headheight}{13.6pt}

% Page geometry
\usepackage[a4paper,top=25mm,bottom=25mm]{geometry}

% Epigraphs
\usepackage{epigraph}
\setlength\epigraphrule{0pt}

% Hyperlinks & References
\usepackage{hyperref}
\hypersetup{
    colorlinks=true,
    linkcolor=blue,
    filecolor=blue,      
    urlcolor=blue,
    citecolor=blue,
}
\urlstyle{same}

% Automatically correct front-side quotes
\usepackage[autostyle=false, style=american]{csquotes}
\MakeOuterQuote{"}

% Graphics
\usepackage{graphicx}
\graphicspath{{Images/}{../Images/}}

% Colour
\usepackage{color}
\usepackage[usenames,dvipsnames]{xcolor}

% SI units
\usepackage{siunitx}

% Microtype goodness
\usepackage{microtype}

% Listings
\usepackage{listings}
\definecolor{backgroundColour}{RGB}{250,250,250}
\definecolor{commentColour}{RGB}{73, 175, 102}
\definecolor{identifierColour}{RGB}{196, 19, 66}
\definecolor{stringColour}{RGB}{252, 156, 30}
\definecolor{keywordColour}{RGB}{50, 38, 224}
\definecolor{lineNumbersColour}{RGB}{127,127,127}
\lstset{ 
  language=Matlab,
  captionpos=b,
  backgroundcolor=\color{backgroundColour},
  basicstyle=\footnotesize,        % the size of the fonts that are used for the code
  breakatwhitespace=false,         % sets if automatic breaks should only happen at whitespace
  breaklines=true,                 % sets automatic line breaking
  postbreak=\mbox{\textcolor{red}{$\hookrightarrow$}\space},
  commentstyle=\color{commentColour},    % comment style
  identifierstyle=\color{identifierColour},
  stringstyle=\color{stringColour},
   keywordstyle=\color{keywordColour},       % keyword style
  %escapeinside={\%*}{*)},          % if you want to add LaTeX within your code
  extendedchars=true,              % lets you use non-ASCII characters; for 8-bits encodings only, does not work with UTF-8
  frame=single,	                   % adds a frame around the code
  keepspaces=true,                 % keeps spaces in text, useful for keeping indentation of code (possibly needs columns=flexible)
  morekeywords={*,...},            % if you want to add more keywords to the set
  numbers=left,                    % where to put the line-numbers; possible values are (none, left, right)
  numbersep=5pt,                   % how far the line-numbers are from the code
  numberstyle=\tiny\color{lineNumbersColour}, % the style that is used for the line-numbers
  rulecolor=\color{black},         % if not set, the frame-color may be changed on line-breaks within not-black text (e.g. comments (green here))
  showspaces=false,                % show spaces everywhere adding particular underscores; it overrides 'showstringspaces'
  showstringspaces=false,          % underline spaces within strings only
  showtabs=false,                  % show tabs within strings adding particular underscores
  stepnumber=1,                    % the step between two line-numbers. If it's 1, each line will be numbered
  tabsize=2,	                   % sets default tabsize to 2 spaces
  %title=\lstname                   % show the filename of files included with \lstinputlisting; also try caption instead of title
}

% Caption stuff
\usepackage{caption}
\usepackage{subcaption}

\makenoidxglossaries

\newacronym{radar}{RADAR}{Radio Detection and Ranging}
\newacronym{dab}{DAB}{Digital Audio Broadcasting}
\newacronym{fm}{FM}{Frequency Modulation}
\newacronym{am}{AM}{Amplitude Modulation}
\newacronym{fdm}{FDM}{Frequency Division Multiplexing}
\newacronym{ofdm}{OFDM}{Orthogonal Frequency Division Multiplexing}
\newacronym{cofdm}{COFDM}{Coded Orthogonal Frequency Division Multiplexing}
\newacronym{dvbt2}{DVB–T2}{Digital Video Broadcasting — Second Generation Terrestrial}
\newacronym{em}{EM}{electromagnetic}
\newacronym{icasa}{ICASA}{Independent Communications Authority of South Africa}
\newacronym{ioo}{IOO}{Illuminators of Opportunity}
\newacronym{pr}{PR}{Passive Radar}
\newacronym{qpsk}{QPSK}{Differential Quadrature Phase-Shift Keying}
\newacronym{dqpsk}{DQPSK}{Differential Quadrature Phase-Shift Keying}
\newacronym{etsi}{ETSI}{European Telecommunications Standards Institute}
\newacronym{psk}{PSK}{Phase Shift Keying}
\newacronym{ask}{ASK}{Amplitude-Shift Keying}
\newacronym{fsk}{FSK}{Frequency-Shift Keying}
\newacronym{iq}{IQ}{In-phase and Quadrature}
\newacronym{prs}{PRS}{Phase Reference Symbol}
\newacronym{dft}{DFT}{Discrete Fourier Transform}
\newacronym{fft}{FFT}{Fast Fourier Transform}
\begin{document}
% ----------------------------------------------------
\chapter{Introduction}
\epigraph{Philosophers have hitherto only interpreted the world in various ways; the point is to change it.}%
    {\emph{---Karl Marx}}
% ----------------------------------------------------

This report describes a project involving the implementation of a processing chain for \gls{dab} signals, specifically oriented towards the context of \gls{pr}. The project consisted of research into the \gls{dab} standard, design and implementation of a corresponding processing chain, and a series of validation steps taken to ensure its correctness. This cursory chapter aims to outline the background for the project, followed by a listing of the project's key objectives, requirements, and scope.

\section{Background}
Fundamentally, radar\footnote{The word \emph{radar} was originally derived from the acronym, \acrlong{radar}; however, it has subsequently become a common noun, and is used in this report as such.} is a fairly straightforward concept, one that is analogous to the biological mechanism of \emph{echolocation}---used by animals such as bats and dolphins. In the simplest form, a radar system consists of a co-located transmitter and receiver. If the transmitter sends out an \gls{em} signal---a short pulse, for example---this signal will propagate through space, hitting objects as it does. The collection of objects is often called a \emph{scene}. Depending on the properties of the objects the \gls{em} waves hit, some of the energy will reflect off them, thus travelling back to the receiver. There will be a variety of time delays between sending out the signal and receiving the reflections, depending on the positions of the objects. Since these waves are travelling at the speed of light, which is known,\footnote{The speed of light in a vacuum, \(c\), is known exactly, as our measurement of distance is actually defined in terms of \(c\), not the other way around. The speed of light in air, however, is slightly slower than this---and is thus an approximation. Nevertheless, this difference is usually negligible.} the time delays can be converted to distances---thus revealing the ranges to the various objects. Additional complexity can be introduced to calculate the objects' actual positions, velocities, and so on. This is the basis for an \emph{active} radar system---because the transmitter is \emph{actively} producing the \gls{em} waves.

A \gls{pr} system, on the other hand, is different: instead of having a transmitter and receiver, it only has the latter. Rather than actively transmitting signals, it uses existing \gls{em} transmissions---sometimes called \gls{ioo}---that are being transmitted by some other operator---a radio station, for example. While the mathematics is slightly more complicated for this approach, the core idea remains the same: reflections of the transmitted signal are measured, and using the calculated time delays, the locations of objects are calculated. This approach is sometimes called Passive Coherent Location, Passive Bistatic Radar, or Commensal Radar. The simple term, \gls{pr}, will be used throughout this report.

For the past century of radar development, much of the focus has been on \emph{active} radar systems, despite \gls{pr} systems appearing as early as 1935. However, in recent decades, the use of \gls{pr} has been shown to have a host of advantages over its active counterpart. As a result, there has been a renewed interest in the field, across a broad range of use cases.

A substantial amount of the research done in \gls{pr} has involved the use of analogue broadcasting signals as \gls{ioo}s, such as \gls{fm} signals---likely because they are so ubiquitous worldwide. Nonetheless, these transmissions have a variety of downsides when implemented in a \gls{pr} system, such as inconsistent range-resolution in the case of \gls{fm}. In comparison, \emph{digital} broadcasting signals avoid many of these issues. If a received signal was modulated using an openly-accessible digital scheme, the signal can be perfectly "reconstructed", by going through a process of demodulation and subsequent remodulation. This procedure removes noise and other interference from the signal, which can result in superior \gls{pr} performance. Furthermore, the use of digital signals as \gls{ioo}s enables one to use a single antenna for the entire \gls{pr} system, compared to the two antennas required for an analogue-based system.

One such digital broadcasting standard, \gls{dab}, has been shown to yield promising results.  However, though work has been published involving the use of \gls{dab} signals in \gls{pr} situations, there are no comprehensive guides for doing so. The design of the \gls{dab} processing chain is usually glossed over in favour of the actual \gls{pr} results. On the other hand, existing \gls{dab}-specific literature focuses too much on the intricacies of the \gls{dab} standard, complicating matters unnecessarily for \gls{pr} applications. Therefore, there is a need for a focused investigation on using \gls{dab} signals in these contexts.

\section{Objectives}
The broad objective of this project was to design and build a \gls{dab} processing chain that could be integrated into a larger \gls{pr} system. The chain was to be tested and validated, and well documented in the form of a report. Note, however, the goal here was not to build a \emph{fully-fledged} \gls{dab} demodulator or decoder---one which could extract the audio signal from a provided \gls{dab} recording, for example. Instead, the intention was to create the components of a \gls{dab} chain that would be required by a \gls{pr} system, and nothing more.
%The chain was expected to work successfully with provided reference data.

\section{System Requirements}
Based on the objectives above, a set of general system requirements was defined. The designed \gls{dab} processing chain was required to have the ability to:

{
    \setstretch{0.8}
    \begin{itemize}
        \item Pre-process a \gls{dab} signal from a provided recording.
        \item Demodulate the frames of the \gls{dab} signal.
        \item Remodulate the demodulated data, back into \gls{dab} frames, but perfectly reconstructed.
    \end{itemize}
}

\section{Scope \& Limitations}
This project was considerably time-bound, with approximately fourteen weeks from its genesis to its submission. Moreover, the ongoing COVID-19 pandemic created additional barriers for both students and staff during this period. Owing to this, the study was intentionally kept fairly narrow in scope.

Formally, the scope of the project centred around an implementation of the required \gls{dab} functionality---pre-processing, demodulation, and remodulation---in \textsc{Matlab}. This code was to be separated into a set of functions, and submitted in conjunction with this report. Motivations for the design decisions made were to be well justified and documented. Block diagrams for each function were also to be documented, as well each function's associated component-steps and algorithms. An array of validation tests was expected to be formulated, including some tests using provided reference data. These tests were to demonstrate convincingly that the chain indeed worked.

In terms of the project's limitations, there were several. Firstly, within the \gls{dab} chain itself, some processing steps could be omitted for simplicity. These being: frequency synchronisation, and the implementation of error correction. Secondly, the \gls{dab} processing chain was not necessarily required to be \emph{optimised}, and its efficiency, execution times, and other related metrics were not to be considered whatsoever. Thirdly, only pre-recorded data was to be used in the demodulation and remodulation chain---in contrast, data would probably be live-streamed in a practical system. Finally, the processing chain was not intended to be integrated into a real-world \gls{pr} system during the course of the project. In fact, no \gls{pr} processing was required to be tested at all.

\section{Report Outline}
This report begins with a broad surveillance of the relevant literature for the project, in Chapter~\ref{ch:literature}. Thereafter, an unpacking of the relevant aspects of the \gls{dab} standard is presented in Chapter~\ref{ch:dab-standard}, including a detailed explanation of the theoretical concepts that underpin the core \gls{dab} functionality. With the standard then established, Chapter~\ref{ch:dab-processing} provides a thorough exhibition of the design of the \gls{dab} processing chain, including the relevant block diagrams and sample outputs. To ensure the correctness of the designed chain, Chapter~\ref{ch:dab-validation} outlines the validation steps that were taken for the project. After this, a short illustrative segment is provided in Chapter~\ref{ch:pr-integration}, showing how the \gls{dab} processing blocks would fit into a larger \gls{pr} system. Finally, conclusions are made and recommendations provided, in Chapters~\ref{ch:conclusions} and~Chapter~\ref{ch:recommendations} respectively. Appendices follow with miscellaneous information that appeared during the report.



% ----------------------------------------------------
\ifstandalone
\bibliography{../Bibliography/References.bib}
\printnoidxglossary[type=\acronymtype,nonumberlist]
\fi
\end{document}
% ----------------------------------------------------
% ----------------------------------------------------
% Literature Review
% ----------------------------------------------------
\documentclass[class=report,11pt,crop=false]{standalone}
% Page geometry
\usepackage[a4paper,margin=25mm,top=25mm,bottom=25mm]{geometry}

% Font choice
\usepackage{lmodern}

% Use IEEE bibliography style
\bibliographystyle{IEEEtran}

% Line spacing
\usepackage{setspace}
\setstretch{1.20}

% Ensure UTF8 encoding
\usepackage[utf8]{inputenc}

% Language standard (not too important)
\usepackage[english]{babel}

% Skip a line in between paragraphs
\usepackage{parskip}

% For the creation of dummy text
\usepackage{blindtext}

% Math
\usepackage{amsmath}

% Header & Footer stuff
\usepackage{fancyhdr}
\pagestyle{fancy}
\fancyhead{}
\fancyhead[R]{\nouppercase{\rightmark}}
\fancyfoot{}
\fancyfoot[C]{\thepage}
\renewcommand{\headrulewidth}{0.0pt}
\renewcommand{\footrulewidth}{0.0pt}
\setlength{\headheight}{13.6pt}

% Page geometry
\usepackage[a4paper,top=25mm,bottom=25mm]{geometry}

% Epigraphs
\usepackage{epigraph}
\setlength\epigraphrule{0pt}

% Hyperlinks & References
\usepackage{hyperref}
\hypersetup{
    colorlinks=true,
    linkcolor=blue,
    filecolor=blue,      
    urlcolor=blue,
    citecolor=blue,
}
\urlstyle{same}

% Automatically correct front-side quotes
\usepackage[autostyle=false, style=american]{csquotes}
\MakeOuterQuote{"}

% Graphics
\usepackage{graphicx}
\graphicspath{{Images/}{../Images/}}

% Colour
\usepackage{color}
\usepackage[usenames,dvipsnames]{xcolor}

% SI units
\usepackage{siunitx}

% Microtype goodness
\usepackage{microtype}

% Listings
\usepackage{listings}
\definecolor{backgroundColour}{RGB}{250,250,250}
\definecolor{commentColour}{RGB}{73, 175, 102}
\definecolor{identifierColour}{RGB}{196, 19, 66}
\definecolor{stringColour}{RGB}{252, 156, 30}
\definecolor{keywordColour}{RGB}{50, 38, 224}
\definecolor{lineNumbersColour}{RGB}{127,127,127}
\lstset{ 
  language=Matlab,
  captionpos=b,
  backgroundcolor=\color{backgroundColour},
  basicstyle=\footnotesize,        % the size of the fonts that are used for the code
  breakatwhitespace=false,         % sets if automatic breaks should only happen at whitespace
  breaklines=true,                 % sets automatic line breaking
  postbreak=\mbox{\textcolor{red}{$\hookrightarrow$}\space},
  commentstyle=\color{commentColour},    % comment style
  identifierstyle=\color{identifierColour},
  stringstyle=\color{stringColour},
   keywordstyle=\color{keywordColour},       % keyword style
  %escapeinside={\%*}{*)},          % if you want to add LaTeX within your code
  extendedchars=true,              % lets you use non-ASCII characters; for 8-bits encodings only, does not work with UTF-8
  frame=single,	                   % adds a frame around the code
  keepspaces=true,                 % keeps spaces in text, useful for keeping indentation of code (possibly needs columns=flexible)
  morekeywords={*,...},            % if you want to add more keywords to the set
  numbers=left,                    % where to put the line-numbers; possible values are (none, left, right)
  numbersep=5pt,                   % how far the line-numbers are from the code
  numberstyle=\tiny\color{lineNumbersColour}, % the style that is used for the line-numbers
  rulecolor=\color{black},         % if not set, the frame-color may be changed on line-breaks within not-black text (e.g. comments (green here))
  showspaces=false,                % show spaces everywhere adding particular underscores; it overrides 'showstringspaces'
  showstringspaces=false,          % underline spaces within strings only
  showtabs=false,                  % show tabs within strings adding particular underscores
  stepnumber=1,                    % the step between two line-numbers. If it's 1, each line will be numbered
  tabsize=2,	                   % sets default tabsize to 2 spaces
  %title=\lstname                   % show the filename of files included with \lstinputlisting; also try caption instead of title
}

% Caption stuff
\usepackage{caption}
\usepackage{subcaption}

\makenoidxglossaries

\newacronym{radar}{RADAR}{Radio Detection and Ranging}
\newacronym{dab}{DAB}{Digital Audio Broadcasting}
\newacronym{fm}{FM}{Frequency Modulation}
\newacronym{am}{AM}{Amplitude Modulation}
\newacronym{fdm}{FDM}{Frequency Division Multiplexing}
\newacronym{ofdm}{OFDM}{Orthogonal Frequency Division Multiplexing}
\newacronym{cofdm}{COFDM}{Coded Orthogonal Frequency Division Multiplexing}
\newacronym{dvbt2}{DVB–T2}{Digital Video Broadcasting — Second Generation Terrestrial}
\newacronym{em}{EM}{electromagnetic}
\newacronym{icasa}{ICASA}{Independent Communications Authority of South Africa}
\newacronym{ioo}{IOO}{Illuminators of Opportunity}
\newacronym{pr}{PR}{Passive Radar}
\newacronym{qpsk}{QPSK}{Differential Quadrature Phase-Shift Keying}
\newacronym{dqpsk}{DQPSK}{Differential Quadrature Phase-Shift Keying}
\newacronym{etsi}{ETSI}{European Telecommunications Standards Institute}
\newacronym{psk}{PSK}{Phase Shift Keying}
\newacronym{ask}{ASK}{Amplitude-Shift Keying}
\newacronym{fsk}{FSK}{Frequency-Shift Keying}
\newacronym{iq}{IQ}{In-phase and Quadrature}
\newacronym{prs}{PRS}{Phase Reference Symbol}
\newacronym{dft}{DFT}{Discrete Fourier Transform}
\newacronym{fft}{FFT}{Fast Fourier Transform}
\begin{document}
\ifstandalone
\tableofcontents
\fi
% ----------------------------------------------------
\chapter{Literature Review}
\epigraph{I say with Didacus Stella, a dwarf standing on the shoulders of a giant may see farther than a giant himself.}%
    {\emph{---Robert Burton}}
% ----------------------------------------------------

\gls{radar} technology has been around for many decades, with early developments occurring in the 1920s and 1930s, advancing rapidly in the years that followed. It, together with many other technologies, developed as a secret military project, and was used extensively in the Second World War. Since then, it has become incredibly ubiquitous, used in a myriad of applications, on both large and small scales. Even the acronym itself has become well-known by the broader public, used today in everyday conversation as a common noun, \emph{radar}.

This chapter hopes to outline a brief history of radar, specifically looking at the development of \emph{passive} radar technology, with its associated advantages and use cases. Following that, there will be a brief look at digital broadcasting technologies. Finally, the literature for using digital broadcasting in passive radar contexts will be reviewed.

% -------- PASSIVE RADAR --------
\section{Radar Concepts}

Fundamentally, radar is a fairly straightforward concept, one that is analogous to the biological mechanism of \emph{echolocation}---used by animals such as bats and dolphins. In the simplest form, a radar system consists of a collocated transmitter and receiver. If the transmitter sends out an \gls{em} signal---a short pulse, for example---this signal will propagate through space, hitting objects as it does. The collection of objects is often called a \emph{scene}. Depending on the properties of the objects the \gls{em} waves hit, some of the energy will reflect off of them, thus travelling back to the receiver. There will be a variety time delays between sending out the signal and receiving the reflections, depending on the positions of the objects. Since these waves are travelling at the speed of light, which is known\footnote{The speed of light in a vacuum, \(c\), is known exactly, as our measurement of distance is actually defined in terms of \(c\), not the other way around. The speed of light in air, however, is slightly slower than this---and is thus an approximation. Nevertheless, this difference is usually negligible.}, the time delays can be converted to distances---thus revealing the range to the various objects. Additional complexity can be introduced to calculate the objects' actual positions, their velocities, and so on. This is the basis for an \emph{active} radar system---because the transmitter is \emph{actively} producing the \gls{em} waves.

A \gls{pr} system, on the other hand, is different---instead of having a transmitter and receiver, it only has the latter. Rather than actively transmitting signals, it uses existing \gls{em} transmissions---sometimes called \gls{ioo}~\cite{Griffiths1992}. While the mathematics is slightly more complicated for this approach, the core idea remains the same---reflections of the transmitted signal are measured, and using the calculated time delays, the locations of objects are calculated.

This section begins with a short history of \gls{pr} and the three possible radar geometries, and then looks specifically at the mechanisms of passive radar and its associated advantages.

\subsection{Brief History}
The history of passive radar naturally begins as a history of radar itself. Like most scientific breakthroughs, it is difficult to attribute the development of radar solely to one individual or group, and frankly it would be disingenuous to do so---it arrived via a long line of successive discoveries, sometimes even with multiple independent discoveries of the same thing~\cite{brown1999technical}. A good starting point for this history is in 1865: when the Scottish mathematical physicist, James Maxwell, published his seminal work, in which the now-famous Maxwell's equations were first demonstrated~\cite{maxwell1865viii}. In doing so, Maxwell predicted that \gls{em} waves travelled through free space at the speed of light~\cite{Sengupta2003}. Three decades later, through a host of experiments done between 1885 and 1889, Heinrich Hertz was able to verify Maxwell's equations and, more relevantly, demonstrate the \emph{reflection} of electromagnetic waves~\cite{hertz1893electromagnetic, Cichon1995}, amongst other things. In the subsequent years, this work was continued by Alexander Stepanovich Popov, Sir Oliver Lodge, Guglielmo Marconi, Nikola Tesla, and others~\cite{Rohling2014, James1989, Griffiths2019}---each of whom helped construct the stage on which radar technology could develop.

Then, in 1904, a 22-year-old German engineer named Christian H\"ulsmeyer filed a patent for the so-called \emph{Telemobiloskop}---which he publicly demonstrated on at least two occasions that year~\cite{Galati2014}. This device was marketed as an anti-collision device, able to detect nearby obstacles using the reflection of \gls{em} waves~\cite{swords1986technical}. Interestingly, the device failed commercially, and there was little interest in it. However, despite its failure, H\"ulsmeyer is today regarded as the father of radar\footnote{Semantically, some scholars have argued that the Telemobiloskop was, in fact, not a RADAR---radio detection and \emph{ranging}---device, as it could not calculate the distance to obstacles, along with some other petty concerns regarding its poor performance~\cite{pritchard1989radar}. Nevertheless, in 2019, H\"ulsmeyer was officially honoured posthumously for his radar contributions, in an IEEE Historic Milestone event~\cite{Griffiths2019}.}. For a thorough history of H\"ulsmeyer and his work, the reader is encouraged to see~\cite{pritchard1989radar} and~\cite{bauer2005christian}.

Around twenty years later, there was a renewed interest in- and rediscovery of radar, independently of H\"ulsmeyer's work. On the other side of the Atlantic in the United States of America, Leo Young and Hoyt Taylor noticed the reflections of high-frequency \gls{em} waves from nearby ships, as these ships moved in between the radio transmitter and receiver.

As retold by Kuschel and O'Hagan in~\cite{kuschel-hagan-history}, the first appearance of a passive radar system occurred in 1935, in the well-known Daventry Experiment. It was here that Sir Robert Watson-Watt and Arnold Wilkins used an existing shortwave BBC transmission for the detection of a plane flying 13 kilometres away. Despite the early appearance of passive radar, interest faded---or, rather, interest shifted to active radar developments.

That was, until a renewed interest arose, such as in Griffiths et al.'s work in~\cite{Griffiths1992}. The authors resurfaced the idea of using existing \gls{em} transmissions as so-called \gls{ioo}.


\subsection{Radar Geometries}

Radar systems can be divided into three broad categories, based on the geometry of their receiver/s and transmitter/s: \emph{monostatic}, \emph{bistatic}, and \emph{multistatic}. Each of these is depicted conceptually in figure~\ref{fig:Radar_Geometry_Depictions}.

\begin{figure}[htbp]
    \centering
    \begin{subfigure}[t]{0.48\textwidth}
        \centering
        \def\svgwidth{\linewidth}
        {\setstretch{0.7} % Line spacing
            \input{../Images/Geometry_Monostatic.pdf_tex}}
        \caption{Monostatic radar geometry}
        \label{fig:Geometry_Monostatic}
    \end{subfigure}%
    ~ 
    \begin{subfigure}[t]{0.48\textwidth}
        \centering
        \def\svgwidth{\linewidth}
        {\setstretch{0.7} % Line spacing
            \input{../Images/Geometry_Bistatic.pdf_tex}}
        \caption{Bistatic radar geometry}
        \label{fig:Geometry_Bistatic}
    \end{subfigure}
    ~
    \begin{subfigure}[t]{0.6\textwidth}
        \centering
        \def\svgwidth{\linewidth}
        {\setstretch{0.7} % Line spacing
            \input{../Images/Geometry_Multistatic.pdf_tex}}
        \caption{Multistatic radar geometry}
        \label{fig:Geometry_Multistatic}
    \end{subfigure}%
    \caption{Conceptual depictions of the three broad radar categories, based on their respective transmitter-receiver geometries}
    \label{fig:Radar_Geometry_Depictions}
\end{figure}

It is clear from figure~\ref{fig:Geometry_Monostatic} that a monostatic geometry is the simplest of the three categories, where the transmitter is collocated with the receiver. This need not be \emph{exactly} so, provided the distance between the transmitter and receiver is much smaller than the distance to the target. However, in most modern situations, the same antenna is used for both transmission and reception, made possible by the duplexer, invented in 1936~\cite{kuschel-hagan-history}.

In contrast, the bistatic geometry, shown in figure~\ref{fig:Geometry_Bistatic}, is slightly more complicated. The geometry itself is straightforward, where the receiver is simply not collocated with the transmitter. However, the mathematics consequently becomes a bit trickier. Suppose the radar system detects that the received signal occurs \(t_d\) seconds after it was transmitted. This can easily be converted to a distance, \(d\), using the speed of light in air, \(c\), where \(d \approx ct_d\). In the monostatic case, this range can be combined with a bearing---the direction in which the antenna is pointing---in order to determine the location of the object. Notice, though, in the bistatic case, knowing only the distance creates an elliptic trace of possible positions, with the transmitter and receiver as the two foci of the ellipse respectively. This can be seen in figure~\ref{fig:Elliptic_Curve}, where two sample target points are shown---notice how \(d = x_1 + y_1 = x_2 + y_2\), even though \(x_1 \ne x_2\) and \(y_1 \ne y_2\). This shows how two completely different locations result in the same reflected time delay.

\begin{figure}
    \centering
    \def\svgwidth{0.7\linewidth}
    {\setstretch{0.7} % Line spacing
        \input{../Images/Elliptic_Curve.pdf_tex}}
    \caption{Elliptic trace of possible target positions as measured by a bistatic radar system}
    \label{fig:Elliptic_Curve}
\end{figure}

As a consequence of this, bistatic radar requires additional measurements in order to locate a target accurately. For example, one can use the doppler shift of the object in order to determine its velocity, and thus pinpoint its position on the elliptic trace. Alternatively, multiple bistatic radar systems can be used simultaneously, and the intersection of the resultant ellipses then indicates the target's location. This latter approach is termed a \emph{multistatic} radar, and is shown in figure~\ref{fig:Geometry_Multistatic}. Note that a multistatic arrangement is loosely defined, and may actually comprise of multiple bistatic and/or monostatic radar systems.

Notice that in passive radar situations, the illuminator of opportunity will never be collocated with the receiver; thus, such systems are always at least bistatic, or are otherwise multistatic. Kuschel and O'Hagan have suggested that the convenience and simplicity of active monostatic radars, especially due to the adoption of the duplexer, shifted focus away from passive radar systems in the years after the Second World War~\cite{kuschel-hagan-history}.

\subsection{Passive Bistatic Radar}

% \subsection{Advantages \& Disadvantages of Passive Radar}
Passive radar has a host of advantages over its active counterpart, as well as a handful of drawbacks.

\subsubsection{Advantages}
Firstly, since passive radar systems use existing transmitters, no additional \gls{em} spectrum allocation is required for their operation. This is particularly relevant in the current technological context, where spectrum is fiercely sought after, often involving huge sums of money and political lobbying. Deploying an active radar system requires licensing for its utilized frequency bands; whereas a passive radar can simply hitchhike off an already-licensed transmission.

Moreover, not only can a passive radar operate without licensing requirements---thus saving costs---it can also operate without the cost of transmitting enormous amounts of power. A useful radar transmitter may need to transmit kilowatts of power, which naturally has an associated electrical cost.

These cost savings make passive radar an interesting case-study in particular for poorer nations. 

One of the major attractions of passive radar is in a military context: since no additional \gls{em} energy is emitted, the receiver location remains covert. Some scholars have shown that this is not completely foolproof, but it certainly poses an advantage over the highly-noticeable active radar systems. Passive radar systems also have a high resistance to EM countermeasures. Also counter-counter-stealth, due to low-observable targets and forward scatter.

\cite{o2009passive}

It is worth noting that passive radar systems are not completely protected against electronic attacks, and there is ongoing research into the possible countermeasures that could be taken against such a system. For example, Giusti et al. have looked at jamming OFDM-based signals~\cite{Giusti2018}, and Sch\"upach and B\"oniger have experimented in specifically jamming DAB signals~\cite{schupbach2016}. For a broad overview of the topic, see~\cite{OHagan2019} by O'Hagan et al. For a more comprehensive look at applying electronic countermeasures to \gls{fm} and \gls{dvbt2} signals, see Paine's work in~\cite{painePHD2019}.

These benefits exist outside of a military domain, too. A case study for this is Peralex Electronics' work done for the Square Kilometer Array (SKA) project. The SKA is a massive radio-telescope. Telescopes are so sensitive that the surrounding zones need to be largely electromagnetically clean within the telescope's operating bands. Unfortunately, planes flying past (and weather radars) fly over and can damage the receivers. However, using an active radar system to detect these planes would be pointless---as the system itself would interfere with the telescopes.

The solution here was using a passive radar system hitchhiking on surrounding FM broadcasts. The FM band falls below the telescope's operating range.

\subsubsection{Disadvantages}

- No control over the transmitter

- Transmitted waveform not designed for radar

- DSI

\subsection{Ambiguity Functions}

% -------- DIGITAL BROADCASTING --------
\section{Digital Broadcasting}
When considering the innumerable technological advancements of the past century, one cannot overlook the enormous impact of wireless broadcasting. The advent of mass communication, first via radio and later via television, has undoubtedly shaped our culture, our politics, and frankly, our entire lives. The ability for many, many people to receive the same message---be it audio, visual, or textual---certainly built the stage for the zeitgeist of the 21st-century.

\subsection{History of Digital Broadcasting}
(Mention the development of RDS?) % TODO

Around the same time that H\"ulsmeyer et al. were experimenting with early radar systems, there was development occurring in wireless \emph{audio} transmissions. In 1900, Reginald Fessenden successfully demonstrated an \gls{am} transmission, over a distance of 1.6km, using his so-called \emph{radiotelephone}. In the decades that followed, radio broadcasting became a staple of (Western) culture. By the 1920s, \gls{am} radios were rolling out to everyday households, and new levels of mass communication were achieved.

All of the investigations hitherto were in analogue broadcasting. The computational power of the early computers was simply inadequate for the demands of digital transmission. Nevertheless, as computers matured, the scene was set for change: investigations into digital broadcasting techniques could begin. For example, in 1985, McNally published a paper entitled, "Digital Audio in Broadcasting,"~\cite{McNally1985}; and in 1988, Weck and Theile laid further foundations with their use of convolutional codes and Viterbi decoding for digital audio broadcasting~\cite{weck1988dab}.

OFDM appears in 1966: \cite{Chang1966}

DFT with OFDM: \cite{Weinstein1971}

Also: \cite{lassalle1987principles}

Also: \cite{Alard1988}

Just before the end of the decade, Rault et al. then released their groundbreaking work on \gls{cofdm}, enabling the leap into digital broadcasting~\cite{Raulta}---with further clarity provided subsequently in~\cite{Shelswell1995} by Shelswell. Shortly after this, a flurry of research into digital broadcasting was released, including digital television in~\cite{Bernard1992, stare1992}, and digital audio in~\cite{shelswell1991, Price1992, Maddocks1992}. Around the same time, work was being done on the "Eureka 147" project in Europe, which later became the \gls{dab} standard---the focus of this report. In the years that followed, many more concepts, designs, and possible architectures were released for digital broadcasting technology.

(Note the importance of OFDM)

Now, in 2020, the technology is far more mature, but global adoption is still mixed. In some states, digital transmissions have completely displaced analog counterparts, such as Germany's nationwide implementation of digital television. In other states, however, progress is slower. In a South African context, the \gls{icasa} promised to shut off analog broadcasts in 2011. At present, this has still not occurred.

Development of Digital TV, see~\cite{Wu2006}

\subsection{Advantages \& Disadvantages of Digital Broadcasting}
The benefits and drawbacks of digital broadcasting depend largely on the particular use case. However, there are a handful of consistent advantages, as well as disadvantages.

\subsubsection{Advantages}
- More efficient bandwidth allocation
- Perfectly decodable video and audio (robust to errors)
- Single Frequency Networks

\subsubsection{Disadvantages}

\subsection{Digital Broadcasting Standards \& Adoption}
There are a host of digital broadcasting standards, spanning a host of use-cases and contexts. To complicate matters further, standards vary by country.

\begin{itemize}
    \item Video standards
        \begin{itemize}
            \item Digital Video Broadcasting (DVB) \\
            This is the most adopted standard, and the most relevant in a South African context.
            \item Advanced Television Systems Committee
        \end{itemize}
    \item Audio standards
        \begin{itemize}
            \item Digital Audio Broadcasting (DAB) \\
            \item 
        \end{itemize}
\end{itemize}

% -------- IMPLEMENTATIONS --------
\section{Digital Broadcasting-based Passive Radar}
With the foundations of both passive radar and digital broadcasting presented, it is now interesting to look at the literature relevant to this project: using digital broadcasting signals in the context of a passive radar chain. When passive radar first started resurfacing in the early 1990s, most---if not all---of the work was being done with \gls{fm} signals. This is logical, as \gls{fm} signals were by-far the most ubiquitous radio format globally at that time. Since then, development has branched out into using other signal modulation types, even using WiFi~\cite{Guo2008} and 

\subsection{Advantages}

Consideration of the ambiguity functions in~\cite{Griffithsa}, and specifically looking at OFDM waveforms in~\cite{Searle2014}. Comprehensive analysis found in~\cite{Malanowski2019}.

\subsection{Existing Implementations}
Much work has been done on using \gls{fm} signals as \gls{ioo}.

O'Hagan et al. in 2007, via a "PBR Demonstrator,"~\cite{OHagan2007}. Even using "Commerical Off-The-Shelf" components, in~\cite{Tong2015}. Detailed look into \gls{fm} implementations, see theses by O'Hagan~\cite{o2009passive}, Tong~\cite{tong2014}, Brown~\cite{brown2013fm}.

One of the earliest works done using digital signals---specifically COFDM signals---was by Poullin in 2005~\cite{Poullin2005}.

Lots of digital video-broadcasting signal implementations, including using DVB-T in~\cite{radmard2012feasibility, palmer2011overview,Peto2014}, and more recently using the newer DVB-T2 in~\cite{OHagan2018} and~\cite{Low2019}.

There have been implementations using \gls{dab}: early on \cite{Yardley2007,Coleman2008}, more recently to detect micro-UAVs in~\cite{Schupbach2017}.
Using signal reconstruction: \cite{ohagan2010dab} instead of conventional DSI suppression.

Comprehensive look at using OFDM signals in passive radar, see~\cite{Berger2010}.

Using multi-illuminator systems, where several \gls{ioo} are combined to leverage benefits of multiple systems. In 2007, Daun and Koch use multistatic~\cite{Daun2007} approach with DAB/DVB-T. Also~\cite{Edrich2014} and~\cite{Paine2018} using FM and DVB-T2.

\section{Summary}

% ----------------------------------------------------
\ifstandalone
\bibliography{../Bibliography/References.bib}
\printnoidxglossary[type=\acronymtype,nonumberlist]
\fi
\end{document}
% ----------------------------------------------------
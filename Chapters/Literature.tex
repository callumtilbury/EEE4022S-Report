% ----------------------------------------------------
% Literature Review
% ----------------------------------------------------
\documentclass[class=report,11pt,crop=false]{standalone}
% Page geometry
\usepackage[a4paper,margin=25mm,top=25mm,bottom=25mm]{geometry}

% Font choice
\usepackage{lmodern}

% Use IEEE bibliography style
\bibliographystyle{IEEEtran}

% Line spacing
\usepackage{setspace}
\setstretch{1.20}

% Ensure UTF8 encoding
\usepackage[utf8]{inputenc}

% Language standard (not too important)
\usepackage[english]{babel}

% Skip a line in between paragraphs
\usepackage{parskip}

% For the creation of dummy text
\usepackage{blindtext}

% Math
\usepackage{amsmath}

% Header & Footer stuff
\usepackage{fancyhdr}
\pagestyle{fancy}
\fancyhead{}
\fancyhead[R]{\nouppercase{\rightmark}}
\fancyfoot{}
\fancyfoot[C]{\thepage}
\renewcommand{\headrulewidth}{0.0pt}
\renewcommand{\footrulewidth}{0.0pt}
\setlength{\headheight}{13.6pt}

% Page geometry
\usepackage[a4paper,top=25mm,bottom=25mm]{geometry}

% Epigraphs
\usepackage{epigraph}
\setlength\epigraphrule{0pt}

% Hyperlinks & References
\usepackage{hyperref}
\hypersetup{
    colorlinks=true,
    linkcolor=blue,
    filecolor=blue,      
    urlcolor=blue,
    citecolor=blue,
}
\urlstyle{same}

% Automatically correct front-side quotes
\usepackage[autostyle=false, style=american]{csquotes}
\MakeOuterQuote{"}

% Graphics
\usepackage{graphicx}
\graphicspath{{Images/}{../Images/}}

% Colour
\usepackage{color}
\usepackage[usenames,dvipsnames]{xcolor}

% SI units
\usepackage{siunitx}

% Microtype goodness
\usepackage{microtype}

% Listings
\usepackage{listings}
\definecolor{backgroundColour}{RGB}{250,250,250}
\definecolor{commentColour}{RGB}{73, 175, 102}
\definecolor{identifierColour}{RGB}{196, 19, 66}
\definecolor{stringColour}{RGB}{252, 156, 30}
\definecolor{keywordColour}{RGB}{50, 38, 224}
\definecolor{lineNumbersColour}{RGB}{127,127,127}
\lstset{ 
  language=Matlab,
  captionpos=b,
  backgroundcolor=\color{backgroundColour},
  basicstyle=\footnotesize,        % the size of the fonts that are used for the code
  breakatwhitespace=false,         % sets if automatic breaks should only happen at whitespace
  breaklines=true,                 % sets automatic line breaking
  postbreak=\mbox{\textcolor{red}{$\hookrightarrow$}\space},
  commentstyle=\color{commentColour},    % comment style
  identifierstyle=\color{identifierColour},
  stringstyle=\color{stringColour},
   keywordstyle=\color{keywordColour},       % keyword style
  %escapeinside={\%*}{*)},          % if you want to add LaTeX within your code
  extendedchars=true,              % lets you use non-ASCII characters; for 8-bits encodings only, does not work with UTF-8
  frame=single,	                   % adds a frame around the code
  keepspaces=true,                 % keeps spaces in text, useful for keeping indentation of code (possibly needs columns=flexible)
  morekeywords={*,...},            % if you want to add more keywords to the set
  numbers=left,                    % where to put the line-numbers; possible values are (none, left, right)
  numbersep=5pt,                   % how far the line-numbers are from the code
  numberstyle=\tiny\color{lineNumbersColour}, % the style that is used for the line-numbers
  rulecolor=\color{black},         % if not set, the frame-color may be changed on line-breaks within not-black text (e.g. comments (green here))
  showspaces=false,                % show spaces everywhere adding particular underscores; it overrides 'showstringspaces'
  showstringspaces=false,          % underline spaces within strings only
  showtabs=false,                  % show tabs within strings adding particular underscores
  stepnumber=1,                    % the step between two line-numbers. If it's 1, each line will be numbered
  tabsize=2,	                   % sets default tabsize to 2 spaces
  %title=\lstname                   % show the filename of files included with \lstinputlisting; also try caption instead of title
}

% Caption stuff
\usepackage{caption}
\usepackage{subcaption}

\makenoidxglossaries

\newacronym{radar}{RADAR}{Radio Detection and Ranging}
\newacronym{dab}{DAB}{Digital Audio Broadcasting}
\newacronym{fm}{FM}{Frequency Modulation}
\newacronym{am}{AM}{Amplitude Modulation}
\newacronym{fdm}{FDM}{Frequency Division Multiplexing}
\newacronym{ofdm}{OFDM}{Orthogonal Frequency Division Multiplexing}
\newacronym{cofdm}{COFDM}{Coded Orthogonal Frequency Division Multiplexing}
\newacronym{dvbt2}{DVB–T2}{Digital Video Broadcasting — Second Generation Terrestrial}
\newacronym{em}{EM}{electromagnetic}
\newacronym{icasa}{ICASA}{Independent Communications Authority of South Africa}
\newacronym{ioo}{IOO}{Illuminators of Opportunity}
\newacronym{pr}{PR}{Passive Radar}
\newacronym{qpsk}{QPSK}{Differential Quadrature Phase-Shift Keying}
\newacronym{dqpsk}{DQPSK}{Differential Quadrature Phase-Shift Keying}
\newacronym{etsi}{ETSI}{European Telecommunications Standards Institute}
\newacronym{psk}{PSK}{Phase Shift Keying}
\newacronym{ask}{ASK}{Amplitude-Shift Keying}
\newacronym{fsk}{FSK}{Frequency-Shift Keying}
\newacronym{iq}{IQ}{In-phase and Quadrature}
\newacronym{prs}{PRS}{Phase Reference Symbol}
\newacronym{dft}{DFT}{Discrete Fourier Transform}
\newacronym{fft}{FFT}{Fast Fourier Transform}
\begin{document}
\ifstandalone
\tableofcontents
\fi
% ----------------------------------------------------
\chapter{Literature Review \label{ch:literature}}
% \epigraph{I say with Didacus Stella, a dwarf standing on the shoulders of a giant may see farther than a giant himself.}%
%     {\emph{---Robert Burton}}
\epigraph{If you wish to make an apple pie from scratch, you must first invent the universe.}%
    {\emph{---Carl Sagan}}
% ----------------------------------------------------
This chapter aims to establish the context for this project, by considering its place within a broader literary sphere. Fundamentally, if one wanted to implement a \gls{dab} processing chain in software, one would only need to consult the \gls{dab} standard document from the \gls{etsi}~\cite{dabstandard}, and perhaps one or two more papers for assistance. However, in doing so, one might erroneously overlook the reasons for which the chain here is being designed: integration into a \gls{pr} system. Consequently, the chain would likely be ill-suited for the specific needs of such a system. Instead, it is important to understand the situations in which this chain is intended to operate, to use later as a foundation for the design, testing, and validation of the chain. Therefore, the surveillance of the literature done here hopes mostly to paint a backdrop for the reader, providing a general insight into the motivations for the project, and the existing work that relates to it.

The chapter begins by examining \gls{pr} and digital broadcasting methods (such as \gls{dab}) in two separate sections, as independent concepts. Each concept has an interesting history, a rich theoretical foundation, a unique set of advantages and disadvantages, and deserves to be considered in isolation. With the stage then set, the  integration of digital broadcasting into \gls{pr} is discussed, including the motivations for such integration, and a broad overview of the work done so far in this domain. Finally, a short critique of the literature is presented, outlining the proposed value of this report.

%---------------------------------
% PASSIVE RADAR
%---------------------------------
\section{Passive Radar}
Fundamentally, radar\footnote{The word \emph{radar} was originanly derived from the acronym, \acrlong{radar}; however, it has subsequently become a common noun, and is used in this report as such.} is a fairly straightforward concept, one that is analogous to the biological mechanism of \emph{echolocation}---used by animals such as bats and dolphins. In the simplest form, a radar system consists of a co-located transmitter and receiver. If the transmitter sends out an \gls{em} signal---a short pulse, for example---this signal will propagate through space, hitting objects as it does. The collection of objects is often called a \emph{scene}. Depending on the properties of the objects the \gls{em} waves hit, some of the energy will reflect off of them, thus travelling back to the receiver. There will be a variety of time delays between sending out the signal and receiving the reflections, depending on the positions of the objects. Since these waves are travelling at the speed of light, which is known,\footnote{The speed of light in a vacuum, \(c\), is known exactly, as our measurement of distance is actually defined in terms of \(c\), not the other way around. The speed of light in air, however, is slightly slower than this---and is thus an approximation. Nevertheless, this difference is usually negligible.} the time delays can be converted to distances---thus revealing the ranges to the various objects. Additional complexity can be introduced to calculate the objects' actual positions, velocities, and so on. This is the basis for an \emph{active} radar system---because the transmitter is \emph{actively} producing the \gls{em} waves.

A \gls{pr} system, on the other hand, is different: instead of having a transmitter and receiver, it only has the latter. Rather than actively transmitting signals, it uses existing \gls{em} transmissions---sometimes called \gls{ioo}~\cite{Griffiths1992}---that are being transmitted by some other operator---a radio station, for example. While the mathematics is slightly more complicated for this approach, the core idea remains the same: reflections of the transmitted signal are measured, and using the calculated time delays, the locations of objects are calculated. This approach is sometimes called Passive Coherent Location, Passive Bistatic Radar, or Commensal Radar. The simple term, \gls{pr}, will be used throughout this report.

\subsection{Brief History}
The history of \gls{pr} naturally begins as a history of radar itself. Like most scientific breakthroughs, it is difficult to attribute the development of radar solely to one individual or group, and frankly, it would be disingenuous to do so---it arrived via a long line of successive discoveries, sometimes even with multiple independent discoveries of the same thing, as highlighted by Brown~\cite{brown1999technical}. A good starting point for this history is in 1865, when the Scottish mathematical physicist, James Maxwell, published his seminal work~\cite{maxwell1865viii}, in which the now-famous \emph{Maxwell's equations} were first demonstrated, albeit in a verbose form. In doing so, Maxwell correctly predicted that \Gls{em} waves travel through free space at the speed of light~\cite{Sengupta2003}. Three decades later, through a host of experiments done between 1885 and 1889, Heinrich Hertz was able to verify these equations and, more relevantly, demonstrate the \emph{reflection} of \gls{em} waves~\cite{hertz1893electromagnetic, Cichon1995}, amongst other things. In the subsequent years, this work was continued by many more scientists---each of whom helped construct the stage on which radar technology could develop.

Then, in 1904, a 22-year-old German engineer named Christian H\"ulsmeyer filed a patent for the so-called \emph{Telemobiloskop}---which he publicly demonstrated on at least two occasions that year~\cite{Galati2014}. This device was marketed as an anti-collision device for ships, able to detect nearby obstacles using the reflection of \gls{em} waves~\cite{swords1986technical}. Interestingly, the device failed commercially, and there was little interest in it; however, despite its failure, H\"ulsmeyer is today regarded as the father of radar.\footnote{Semantically, some scholars have argued that the \emph{Telemobiloskop} was, in fact, not a \acrshort{radar}---radio detection and \emph{ranging}---device, as it could not calculate the distance to obstacles, along with some other petty concerns regarding its poor performance~\cite{pritchard1989radar}. Nevertheless, in 2019, H\"ulsmeyer was officially honoured posthumously for his radar contributions, in an IEEE Historic Milestone event~\cite{Griffiths2019}.} For a thorough history of H\"ulsmeyer and his achievements, the reader is encouraged to see~\cite{pritchard1989radar} and~\cite{bauer2005christian}.

Some twenty years later, after a period with no tangible developments of H\"ulsmeyer's work, the concept of radar was essentially rediscovered independently, on the other side of the Atlantic, in the United States. Leo Young and Hoyt Taylor, two radio engineers at the U.S. Naval Research Laboratory, noticed the reflections of high-frequency \gls{em} waves from nearby ships, as these ships moved in between a radio transmitter and receiver~\cite{brown1999technical}. More time passed before further developments occurred; but, eventually, fully-fledged radar systems began to emerge.

The first appearance of a \emph{passive} radar system, as retold by Kuschel and O'Hagan in~\cite{kuschel-hagan-history}, occurred in~1935, in the well-known Daventry Experiment. It was here that Sir Robert Watson-Watt and Arnold Wilkins used an existing shortwave BBC transmission for the detection of a plane flying 13~kilometres away. However, despite the early appearance of \gls{pr}, interest primarily shifted to active radar developments for the following decades. That was until a renewed interest arose in the mid-1980s, in Griffiths et al.'s work~\cite{Griffiths1986}. It was here that the authors resurfaced the idea of using existing \gls{em} transmissions as so-called \gls{ioo}, for \gls{pr} detection. Since then, the field has further grown, and continues to grow still.

\subsection{Radar Geometries}
Radar systems can be divided into three broad categories, based on the geometry of their receiver/s and transmitter/s: \emph{monostatic}, \emph{bistatic}, and \emph{multistatic}. Each of these is depicted conceptually in Figure~\ref{fig:Radar_Geometry_Depictions}.

\begin{figure}[htbp]
    \centering
    \captionsetup{type=figure}
    \begin{subfigure}[t]{0.48\textwidth}
        \centering
        \def\svgwidth{\linewidth}
        {\setstretch{0.7} % Line spacing
            \input{../Images/Geometry_Monostatic.pdf_tex}}
        \caption{Monostatic radar geometry}
        \label{fig:Geometry_Monostatic}
    \end{subfigure}%
    ~ 
    \begin{subfigure}[t]{0.48\textwidth}
        \centering
        \def\svgwidth{\linewidth}
        {\setstretch{0.7} % Line spacing
            \input{../Images/Geometry_Bistatic.pdf_tex}}
        \caption{Bistatic radar geometry}
        \label{fig:Geometry_Bistatic}
    \end{subfigure}
    ~
    \begin{subfigure}[t]{0.6\textwidth}
        \centering
        \def\svgwidth{\linewidth}
        {\setstretch{0.7} % Line spacing
            \input{../Images/Geometry_Multistatic.pdf_tex}}
        \caption{Multistatic radar geometry}
        \label{fig:Geometry_Multistatic}
    \end{subfigure}%
    \caption{Conceptual depictions of the three broad radar categories, based on their respective transmitter-receiver geometries.}
    \label{fig:Radar_Geometry_Depictions}
\end{figure}

It is clear from Figure~\ref{fig:Geometry_Monostatic} that a monostatic geometry is the simplest of the three categories, where the transmitter is co-located with the receiver. This need not be \emph{exactly} so, provided the distance between the transmitter and receiver is much smaller than the distance to the target. However, in most modern situations, the same antenna is used for both transmission and reception, made possible by the duplexer, invented in 1936~\cite{kuschel-hagan-history}.

In contrast, the bistatic geometry, shown in Figure~\ref{fig:Geometry_Bistatic}, is slightly more complicated. The geometry itself is straightforward, where the receiver is simply not co-located with the transmitter; however, the mathematics consequently becomes more complicated. To illustrate, suppose the radar system detects that the received signal occurs~\(t_d\) seconds after it was transmitted. This time delay can easily be converted to the distance that the \gls{em} wave travelled, \(d\), using the speed of light,~\(c\), where~\(d \approx ct_d\). In a monostatic case, this range can be combined with a bearing---the direction in which the antenna is pointing---in order to determine the location of the object. Notice, though, in the bistatic case, knowing only the distance creates an elliptic trace of possible positions---sometimes called an \emph{isorange contour}---with the transmitter and receiver as the two foci of the ellipse respectively. This situation can be seen in Figure~\ref{fig:Elliptic_Curve}, where two possible target points are shown as examples. Of course, the actual target could be anywhere on the contour.

\begin{figure}
    \centering
    \captionsetup{type=figure}
    \def\svgwidth{0.5\linewidth}
    {\scriptsize
        \setstretch{0.7} % Line spacing
        \input{../Images/Elliptic_Curve.pdf_tex}}
    \caption{Elliptic trace of possible target positions as measured by a bistatic radar system.}
    \label{fig:Elliptic_Curve}
\end{figure}

Notice in the figure that \(d = x_1 + x_2\) and \(d = y_1 + y_2\), even though \(x_1 \ne y_1\) and \(x_2 \ne y_2\). This demonstrates an example of how two completely different locations can result in the same reflected time delay. As a consequence of this, bistatic radars require additional measurements in order to locate a target accurately. For example, one can use the doppler shift of the object in order to determine its velocity, and thus pinpoint its position on the elliptic trace.\footnote{For a thorough description of the signal processing required for Passive Bistatic Radar systems, see~\cite{Malanowski2019}.} Alternatively, multiple bistatic radar systems can be used simultaneously, and the intersection of the resultant ellipses then indicates the target's location. This latter approach is termed a \emph{multistatic} radar, and was depicted previously in Figure~\ref{fig:Geometry_Multistatic}. Note that a multistatic arrangement is loosely defined, and may actually comprise multiple bistatic and/or monostatic radar systems. 

Intuitively, in \gls{pr} situations, it is highly unlikely that the \gls{ioo}'s transmitter will be co-located with the system's receiver; thus, such systems are almost always at least bistatic, or are otherwise multistatic. Interestingly, Kuschel and O'Hagan have suggested that the convenience and simplicity of monostatic radars---enabled largely by the invention of the duplexer---shifted focus away from \gls{pr} technology in the early years of radar~\cite{kuschel-hagan-history}.

\subsection{Advantages \& Disadvantages of Passive Radar}
The use of \gls{pr} is attractive for a variety of reasons, hence the renewed interest therein over the past three decades. Many scholars have discussed these advantages comprehensively, and they are summarised nicely by O'Hagan~\cite{o2009passive}, whose reasons will be briefly mentioned here. Fundamentally, a \gls{pr} system is advantageous because, by definition, it does not use or need its own transmitter---it simply hitchhikes off existing transmissions. Several benefits arise from this simple fact.

Firstly, transmitting \gls{em} signals requires energy, and such energy is not free. A useful active radar transmitter may need to transmit kilowatts of power when operating, which naturally has an associated electrical cost. Licensing for the usage of the \gls{em} spectrum is also not cheap---especially not in recent years, where there has been increasingly-fierce competition and political lobbying for spectrum allocation. Moreover, the antennas used for these transmissions are often required to be bulky. Therefore, a \gls{pr} system can be both cheaper to operate and maintain, and more compact than an active counterpart.

In a military context, the lack of a transmitter also enables \emph{covertness} of the system. An active radar can be easily detected by an enemy via the radiation that is emitted, and thus puts it at risk of sabotage by \gls{ecm} or otherwise. A \gls{pr} system, on the other hand, is far more difficult to detect, and is therefore less susceptible to such attacks. Furthermore, O'Hagan points out that the use of the VHF frequency band---which is not allocated for radar purposes---in a \gls{pr} system enables certain counter-stealth benefits for so-called \emph{low-observable} targets~\cite{o2009passive}.

Another interesting advantage is demonstrated in a real-world use-case. Peralex Electronics, along with work done by Paine et al., deployed a \gls{pr} system at one of the \gls{ska} sites in South Africa~\cite{Paine2018}. The \gls{ska} is a massive radio-telescope, which is so extremely sensitive that the zones surrounding it need to be electromagnetically "clean" within the telescope's operating bands. Unfortunately, small aircraft regularly fly over the site, thereby introducing \gls{em} interference from their weather radars, transponders, and other electrical equipment. It is vital that the telescope does not operate during these instances. However, using an \emph{active} radar system to detect these aircraft would be pointless, as the radar system itself would then interfere with the telescope. The solution here was to use a \gls{pr} system that hitchhikes on surrounding \gls{fm} broadcasts. Since the \gls{fm} spectrum falls below the telescope's operating band, the radar can detect incoming aircraft, while the telescope continues to operate normally.

Naturally, some disadvantages exist for \gls{pr} systems, which are also discussed in~\cite{o2009passive}. Ideally, the waveforms used in a radar system's transmitter should have certain characteristics that facilitate detection in its receiver. But, in a \gls{pr} context, one has no control over the content that is broadcast on a particular \gls{ioo}, possibly making the system less reliable, depending on the \gls{ioo} used. Furthermore, \gls{pr} systems can also suffer from \gls{dsi}---a scenario in which the signal travelling directly from the transmitter to the receiver overpowers the reflected signals from the scene, thereby limiting the radar's "sensitivity".

It is also important to note that \gls{pr} systems do not have outright \emph{immunity} against \gls{ecm}. In fact, there is ongoing research into the possible electronic attacks that can be taken against such a system. For example, Giusti et al. have looked at jamming \gls{ofdm}-based signals~\cite{Giusti2018}, and Sch\"upach and B\"oniger have experimented in specifically jamming \gls{dab} signals~\cite{schupbach2016}, both in \gls{pr} contexts. For a broad overview of this topic, see~\cite{OHagan2019} by O'Hagan et al. For a more comprehensive look at applying \gls{ecm} specifically to \gls{fm} and \gls{dvbt2} signals, see Paine's work in~\cite{painePHD2019}.

%---------------------------------
% DIGITAL BROADCASTING
%---------------------------------
\section{Digital Broadcasting}
When considering the innumerable technological advancements of the past century, one cannot overlook the enormous impact of wireless broadcasting. The advent of mass communication, first via radio and later via television, has undoubtedly shaped our culture, our politics, and frankly, our entire lives. \emph{Digital} broadcasting, in contrast to \emph{analogue} broadcasting, involves the use of digital signals for the modulation of the carrier waves in a broadcast. Of course, the \gls{em} signals that are actually transmitted are themselves analogue in nature, but the information encoded in them is discretized to a finite set of possible values.

Understanding the development of digital broadcasting---why it has or has not been adopted in its various instances---is an important question in the context of \gls{pr}, for the following reason: a \gls{pr} system that uses a digital broadcast as an \gls{ioo} relies entirely on that digital broadcast existing in perpetuity. When a particular broadcasting standard ceases to be relevant, the systems designed around that standard become useless. Therefore, it is vital to gauge the availability of a particular standard to contemplate the development effort required to integrate it into a practical \gls{pr} system.

This section attempts to show briefly the history of and motivations for digital broadcasting, \textcolor{red}{as well as the relevant standards that presently exist in the world}. Since this report focuses specifically on \gls{dab}, the entire chapter following this one is dedicated to the \emph{theory} behind the \gls{dab} standard and its associated concepts. To avoid needless repetition, these concepts will not be covered comprehensively here.

\subsection{Brief History}
Around the same time that H\"ulsmeyer et al. were experimenting with early radar systems, there was development occurring in wireless \emph{audio} transmissions. For example, in the final days of 1906, Reginald Fessenden successfully demonstrated the first wireless broadcast by wishing a Merry Christmas to nearby ships of the U.S. Navy, as Belrose reminisces in~\cite{Belrose2012}. In the decades that followed, radio broadcasting became a staple of society.

For most of the twentieth century, radio was operated solely via analogue broadcasting techniques, firstly with \gls{am}, and later with \gls{fm} too. The theoretical foundations for digital broadcasting did exist for some time, arguably beginning with Chang's seminal work on \gls{ofdm} in 1966~\cite{Chang1966}, followed by Weinstein and Ebert's insight into using the \gls{dft} for \gls{ofdm} in 1971~\cite{Weinstein1971}. However, the computational power of the computers at the time was simply inadequate for the demands of a digital modulation scheme. Nonetheless, as computers matured, the scene was set to change; investigations into digital broadcasting techniques could begin. For example, in 1985, McNally wrote an article entitled, "Digital Audio in Broadcasting," for the IEEE's Acoustics, Speech, and Signal Processing magazine~\cite{McNally1985}. Two years later, Alard and Lassalle~\cite{lassalle1987principles, Alard1988} discussed the modulation and channel coding requirements for an \gls{ofdm} system; and in 1988, Weck and Theile~\cite{weck1988dab} laid further foundations with their proposal to use convolutional codes and Viterbi decoding in such contexts. Just before the end of the decade, Rault et al.~\cite{Raulta} then released their work on \gls{cofdm}, which enabled robust mitigation of multipath effects in a mobile digital receiver.

Naturally, these developments spurred on a flurry of research into digital broadcasting, for both television~\cite{Bernard1992, stare1992}, and audio~\cite{shelswell1991, Price1992, Maddocks1992} purposes. Around the same time, work was being done on the \emph{Eureka~147} project~\cite{Halliera} in Europe, which later became the \gls{dab} standard---the focus of this report. In the years that followed, many more concepts, designs, and possible architectures were released for digital broadcasting technology. For a richer history of digital radio, specifically looking at \gls{dab}, see~\cite{ONeill2009}; for an history of digital television, see~\cite{Wu2006}.

\subsection{Advantages \& Disadvantages of Digital Broadcasting}
The benefits and drawbacks of digital broadcasting are somewhat dependent on the particular use-case in question. Nonetheless, there are a handful of consistent advantages when using a digital scheme compared to an analogue equivalent. Gandy~\cite{gandy2003dab}, and Witherow and Laven~\cite{Witherow1995}---both on behalf of the BBC---highlight some of the upsides of a \gls{dab} system, though many of these can be extended to other modes of digital broadcasting too. A handful of them will be highlighted below.

One of the most useful features of digital broadcasting is the spectral efficiency that it can achieve. As mentioned previously in the motivations for \gls{pr}, the \gls{em} spectrum is a fiercely contested resource. By enabling more content to be transmitted using the same bandwidth, these broadcasting techniques can ease the pressure on regulators, bring down the cost per programme, and ultimately, provide more options for the consumer. For example, Gandy considers the situation in the United Kingdom, stating that \emph{six} audio channels in a \gls{dab} broadcast can fit into the bandwidth of a single \gls{fm} broadcast. Note, too, that the actual \emph{quality} of the digital signals is also far superior. \gls{dab} can achieve high-quality audio, with a fidelity comparable to the compact disc. Digital television, too, can achieve high-definition broadcasts.

Digital broadcasts are also more robust in the presence of multipath---which is a massive challenge in \gls{fm} radio. Various error-correction techniques and other protective mechanisms---such as guard intervals in \gls{ofdm} transmissions---can be used with great success. This further enables the use of \gls{sfn}, where nearby transmitters can use the same carrier frequencies for their transmissions. Consequently, these broadcasts can be even more efficient when considering the spectra and power they require.

Finally, digital broadcasting also offers more functionality to the end-user. A \gls{dab} signal, for example, can include a multitude of media types other than audio, including basic graphics, textual information for traffic or station updates, and so on~\cite{dabstandard}.

Despite the many benefits of digital broadcasting systems---of which only some were mentioned---analogue broadcasting systems remain popular around the world. This reality is less about the \emph{disadvantages} of moving towards digital broadcasting, and more about the practicalities of doing so. Since analogue broadcasting has existed for many decades, the receivers for \gls{am}, \gls{fm}, and similar transmissions remain cheaply and widely available. Accordingly, the transition from analogue to digital requires an enormous amount of political will and regulatory effort, as demonstrated by South Africa's failed switch-over. Here, the \emph{Broadcasting Digital Migration Policy} was issued by the Department of Communications in 2008~\cite{icasaNotice2008}, where the \gls{icasa} was tasked with rolling-out digital television, such that there would be an "analogue switch-off" in 2011. As of late-2020, after countless delays, this has yet to occur.

\subsection{Current Standards \& Adoption}
% The current digital broadcasting landscape is somewhat divided across the globe. For digital television,  For digital audio transmissions---superseding \gls{am} and \gls{fm}---

%---------------------------------
% DB-based PR
%---------------------------------
\section{Digital Broadcasting-based Passive Radar}
% \subsection{Overview}
With the foundations of both \gls{pr} and digital broadcasting presented, it is now interesting to look at the use of the latter in the former. That is, the use of digital broadcasting signals as the \gls{ioo}s for a \gls{pr} system.

\subsection{Motivation}
When Griffiths et al. first resurfaced the idea of \gls{pr}, their work was done using analogue television signals---both terrestrial~\cite{Griffiths1986} and satellite-borne~\cite{Griffiths1992}---as the \gls{ioo}s. As the field grew, however, the majority of the research focused on using \gls{fm} signals instead. This makes sense, considering the ubiquity of \gls{fm} radio across the globe. Consequently, much of the existing work has been done in this domain. For a straightforward example, consider~\cite{OHagan2007}; for a more thorough elucidation, see the dissertations prepared by O'Hagan~\cite{o2009passive}, Brown~\cite{brown2013fm}, and Tong~\cite{tong2014} respectively. The present state of \gls{fm}-based \gls{pr} is fairly advanced, and some authors have even demonstrated successful systems that solely use readily-available, "commercial off-the-shelf" components, such as in~\cite{Tong2015} and~\cite{Moser2019}.

Using \gls{fm} signals as \gls{ioo}s has some downsides, though. It can be shown that the range resolution of a radar system is a function of the bandwidth of the signal used for detection. In turn, since the bandwidth of an \gls{fm} signal is time-variant---it is a function of its programme content---the range resolution of a \gls{fm}-based \gls{pr} system is also time-variant. This naturally results in unreliable system performance~\cite{ohagan2010dab}. Moreover, O'Hagan~\cite{OHagan2007} points out that \gls{fm}-based systems still have to overcome the major problem of \gls{dsi}. While various suppression techniques do exist, the \gls{fm} signal itself offers no assistance.

Digital signals, on the other hand, have superior performance in these dimensions where \gls{fm} signals fall short. For example, \gls{cofdm}-based signals---such as \gls{dab} and \gls{dvbt2}---have time-\emph{invariant} bandwidths, regardless of the instantaneous programme content that they are transmitting. Consequently, such signals facilitate a reliable range-resolution when used as \gls{ioo}s for \gls{pr}~\cite{ohagan2010dab}. Griffiths and Baker~\cite{Griffithsa} have also shown definitively, for similar reasons, that the ambiguity functions for digitally-modulated signals are far more auspicious than those for analogue signals. This implies that such signals are well-suited for radar applications. Specific analysis of the ambiguity functions for \gls{ofdm} signals has also been done in~\cite{Searle2014}.

Furthermore, digital broadcasts have another important advantage. Since digital broadcasts usually adhere to an openly-accessible standard---such as \gls{etsi}'s \gls{dab} standard~\cite{dabstandard}---a received signal can be demodulated, and then \emph{remodulated}, thereby creating a perfect reconstruction of the received signal---without noise and interference. In doing so, the requirements for the suppression of \gls{dsi} become less stringent. This approach was put forward in~\cite{ohagan2010dab}, and is also discussed in~\cite{Berger2010}.

This "perfect reconstruction" method has also been taken one step further. A conventional \gls{pr} system requires two signals: a \emph{reference} signal, which is a measurement aimed directly at the original transmitter, and a \emph{surveillance} signal, which is a measurement aimed at the scene to record the various reflections of the transmitted signal. Usually, these signals must be recorded by two independent antennas. However, as proposed by Searle et al.~\cite{Searle2015}, digital broadcasting-based \gls{pr} systems can generate the reference data \emph{from} the surveillance data, by reconstructing it perfectly.\footnote{This approach was also put forward in~\cite{Barott2014}, where the \gls{atsc} standard---a non-\gls{ofdm} digital television approach---was used.} This is massively advantageous, and forms the central motivation for this project.

\subsection{Existing Implementations}
There have been many implementations of digital broadcasting-based \gls{pr} described in the literature, beginning with Poullin's work~\cite{Poullin2005} back in 2005, which dealt with the use of \gls{cofdm} transmissions in \gls{pr}. Since then, numerous papers have used digital \emph{television} signals as \gls{ioo}s, probably because of the widespread adoption of digital television around the world. Earlier works used the older \gls{dvbt} standard, such as in~\cite{radmard2012feasibility, palmer2011overview,Peto2014}; but, more recently, the newer \gls{dvbt2} standard has also been used~\cite{OHagan2018, Low2019}.

There have been implementations using \gls{dab} too, such as from Yardley and Coleman~\cite{Yardley2007, Coleman2008}. More recently, Sch\"upach et al. have taken on various \gls{dab}-based \gls{pr} projects~\cite{Schupbach2017, Moser2019}.

Some progress has also been made using \emph{multi-illuminator} systems, where several \gls{ioo}s are used simultaneously, to leverage the benefits of multiple approaches at once. For example, Daun and Koch~\cite{Daun2007} tested a multistatic approach that used \gls{dab} and \gls{dvbt} signals together. See also~\cite{Edrich2014} and~\cite{Paine2018}, for work done using \gls{fm} and \gls{dvbt2} together.

\section{Literature Critique}
It has been clear from this chapter that a comprehensive amount of literature exists on the topics of \gls{pr}, digital broadcasting, and the integration of the two. Much has already been written about the motivations for these fields and their theoretical underpinnings. In the case of \gls{dab}, papers and books have demonstrated successful receiver architectures, outlined audio extraction techniques, and considered error-correction thoroughly. In the case of \gls{pr}, many real-world tests have been done that consider accuracy, efficiency, reliability, and more. Furthermore, \gls{dab}-based \gls{pr} systems have also been demonstrated and tested.

However, the literature lacks a thorough exposition of the processing chain required for the use of \gls{dab} signals in \gls{pr}. Existing \gls{pr} work omits many details regarding the design choices made for such systems, and existing \gls{dab} work focuses too much on the data perspective---the playback of audio, the processing of incoming traffic information, etc. This report, then, hopes to unpack the \gls{dab} standard insofar as a \gls{pr} context would require, and thereafter, detail the processing chain design for \gls{dab}-based \gls{pr} applications.

% ----------------------------------------------------
\ifstandalone
\bibliography{../Bibliography/References.bib}
\printnoidxglossary[type=\acronymtype,nonumberlist]
\fi
\end{document}
% ----------------------------------------------------
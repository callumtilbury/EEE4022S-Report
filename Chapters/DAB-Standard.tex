% ----------------------------------------------------
% DAB Standard
% ----------------------------------------------------
\documentclass[class=report,11pt,crop=false]{standalone}
\input{../Style/ChapterStyle.tex}
\makenoidxglossaries

\newacronym{radar}{RADAR}{Radio Detection and Ranging}
\newacronym{dab}{DAB}{Digital Audio Broadcasting}
\newacronym{fm}{FM}{Frequency Modulation}
\newacronym{am}{AM}{Amplitude Modulation}
\newacronym{fdm}{FDM}{Frequency Division Multiplexing}
\newacronym{ofdm}{OFDM}{Orthogonal Frequency Division Multiplexing}
\newacronym{cofdm}{COFDM}{Coded Orthogonal Frequency Division Multiplexing}
\newacronym{dvbt2}{DVB–T2}{Digital Video Broadcasting — Second Generation Terrestrial}
\newacronym{em}{EM}{electromagnetic}
\newacronym{icasa}{ICASA}{Independent Communications Authority of South Africa}
\newacronym{ioo}{IOO}{Illuminators of Opportunity}
\newacronym{pr}{PR}{Passive Radar}
\newacronym{qpsk}{QPSK}{Quadrature Phase-Shift Keying}
\newacronym{dqpsk}{DQPSK}{Differential~Quadrature~Phase-Shift~Keying}
\newacronym{etsi}{ETSI}{European Telecommunications Standards Institute}
\newacronym{psk}{PSK}{Phase Shift Keying}
\newacronym{ask}{ASK}{Amplitude-Shift Keying}
\newacronym{fsk}{FSK}{Frequency-Shift Keying}
\newacronym{iq}{IQ}{In-phase and Quadrature}
\newacronym{ns}{NS}{Null Symbol}
\newacronym{prs}{PRS}{Phase Reference Symbol}
\newacronym{fic}{FIC}{Fast Information Channel}
\newacronym{msc}{MSC}{Main Service Channel}
\newacronym{dft}{DFT}{Discrete Fourier Transform}
\newacronym{idft}{IDFT}{Inverse Discrete Fourier Transform}
\newacronym{fft}{FFT}{Fast Fourier Transform}
\newacronym{ifft}{IFFT}{Inverse Fast Fourier Transform}
\newacronym{fec}{FEC}{Forward Error Correction}
\newacronym{ard}{ARD}{Amplitude-Range-Doppler}
\newacronym{snr}{SNR}{Signal-to-Noise Ratio}
\newacronym{isi}{ISI}{Intersymbol Interference}
\newacronym{mcm}{MCM}{Multicarrier Modulation}
\begin{document}
% ----------------------------------------------------
\chapter{Digital Audio Broadcasting: Standard}
\epigraph{Where the waters do agree, it is quite wonderful the relief they give.}%
{\emph{Jane Austen, Emma}}
% ----------------------------------------------------

\section{Overview}
The Digital Audio Broadcasting standard~\cite{dabstandard}.

\section{Orthogonal Frequency Division Multiplexing}
A core component of digital broadcasting rests in \acrfull{ofdm}. This multiplexing scheme is used in a variety of contexts, particularly in other emerging technologies such as 4G and 5G cellular networks, and others.

\subsection{Motivation}
Suppose one wanted to transmit a set of data, \(\mathcal{D}\), using some arbitrary modulation scheme. There are, in essence, two approaches to do this. The first, which is markedly simpler, is a serial approach: to modulate the carrier wave with successive elements of \(\mathcal{D}\). For example, if one was using a straightforward amplitude modulation scheme, as shown in figure ().

Alternatively, one could take a parallel approach. That is, to split \(\mathcal{D}\) into \(n\) subsets, and use multiple channels. This is the essence of frequency division multiplexing.

\subsection{Frequency Division Multiplexing}



\subsection{Orthogonal Carriers}


\section{Phase Shift Keying}
\subsection{Quadrature Phase Shift Keying}
\subsection{Differential Quadrature Phase Shift Keying}


\section{Transmission Frame}


\subsection{Null Symbol}


\subsection{Phase Reference Symbol}


\subsection{Fast Information Channel}


\subsection{Main Service Channel}


\subsection{Guard Intervals}


\section{Forward Error Correction}


\section{Summary}


% ----------------------------------------------------
\ifstandalone
\bibliography{../Bibliography/References.bib}
\printnoidxglossary[type=\acronymtype,nonumberlist]
\fi
\end{document}
% ----------------------------------------------------
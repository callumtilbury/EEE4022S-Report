% ----------------------------------------------------
% Conclusions
% ----------------------------------------------------
\documentclass[class=report,11pt,crop=false]{standalone}
% Page geometry
\usepackage[a4paper,margin=25mm,top=25mm,bottom=25mm]{geometry}

% Font choice
\usepackage{lmodern}

% Use IEEE bibliography style
\bibliographystyle{IEEEtran}

% Line spacing
\usepackage{setspace}
\setstretch{1.20}

% Ensure UTF8 encoding
\usepackage[utf8]{inputenc}

% Language standard (not too important)
\usepackage[english]{babel}

% Skip a line in between paragraphs
\usepackage{parskip}

% For the creation of dummy text
\usepackage{blindtext}

% Math
\usepackage{amsmath}

% Header & Footer stuff
\usepackage{fancyhdr}
\pagestyle{fancy}
\fancyhead{}
\fancyhead[R]{\nouppercase{\rightmark}}
\fancyfoot{}
\fancyfoot[C]{\thepage}
\renewcommand{\headrulewidth}{0.0pt}
\renewcommand{\footrulewidth}{0.0pt}
\setlength{\headheight}{13.6pt}

% Page geometry
\usepackage[a4paper,top=25mm,bottom=25mm]{geometry}

% Epigraphs
\usepackage{epigraph}
\setlength\epigraphrule{0pt}

% Hyperlinks & References
\usepackage{hyperref}
\hypersetup{
    colorlinks=true,
    linkcolor=blue,
    filecolor=blue,      
    urlcolor=blue,
    citecolor=blue,
}
\urlstyle{same}

% Automatically correct front-side quotes
\usepackage[autostyle=false, style=american]{csquotes}
\MakeOuterQuote{"}

% Graphics
\usepackage{graphicx}
\graphicspath{{Images/}{../Images/}}

% Colour
\usepackage{color}
\usepackage[usenames,dvipsnames]{xcolor}

% SI units
\usepackage{siunitx}

% Microtype goodness
\usepackage{microtype}

% Listings
\usepackage{listings}
\definecolor{backgroundColour}{RGB}{250,250,250}
\definecolor{commentColour}{RGB}{73, 175, 102}
\definecolor{identifierColour}{RGB}{196, 19, 66}
\definecolor{stringColour}{RGB}{252, 156, 30}
\definecolor{keywordColour}{RGB}{50, 38, 224}
\definecolor{lineNumbersColour}{RGB}{127,127,127}
\lstset{ 
  language=Matlab,
  captionpos=b,
  backgroundcolor=\color{backgroundColour},
  basicstyle=\footnotesize,        % the size of the fonts that are used for the code
  breakatwhitespace=false,         % sets if automatic breaks should only happen at whitespace
  breaklines=true,                 % sets automatic line breaking
  postbreak=\mbox{\textcolor{red}{$\hookrightarrow$}\space},
  commentstyle=\color{commentColour},    % comment style
  identifierstyle=\color{identifierColour},
  stringstyle=\color{stringColour},
   keywordstyle=\color{keywordColour},       % keyword style
  %escapeinside={\%*}{*)},          % if you want to add LaTeX within your code
  extendedchars=true,              % lets you use non-ASCII characters; for 8-bits encodings only, does not work with UTF-8
  frame=single,	                   % adds a frame around the code
  keepspaces=true,                 % keeps spaces in text, useful for keeping indentation of code (possibly needs columns=flexible)
  morekeywords={*,...},            % if you want to add more keywords to the set
  numbers=left,                    % where to put the line-numbers; possible values are (none, left, right)
  numbersep=5pt,                   % how far the line-numbers are from the code
  numberstyle=\tiny\color{lineNumbersColour}, % the style that is used for the line-numbers
  rulecolor=\color{black},         % if not set, the frame-color may be changed on line-breaks within not-black text (e.g. comments (green here))
  showspaces=false,                % show spaces everywhere adding particular underscores; it overrides 'showstringspaces'
  showstringspaces=false,          % underline spaces within strings only
  showtabs=false,                  % show tabs within strings adding particular underscores
  stepnumber=1,                    % the step between two line-numbers. If it's 1, each line will be numbered
  tabsize=2,	                   % sets default tabsize to 2 spaces
  %title=\lstname                   % show the filename of files included with \lstinputlisting; also try caption instead of title
}

% Caption stuff
\usepackage{caption}
\usepackage{subcaption}

\makenoidxglossaries

\newacronym{radar}{RADAR}{Radio Detection and Ranging}
\newacronym{dab}{DAB}{Digital Audio Broadcasting}
\newacronym{fm}{FM}{Frequency Modulation}
\newacronym{am}{AM}{Amplitude Modulation}
\newacronym{fdm}{FDM}{Frequency Division Multiplexing}
\newacronym{ofdm}{OFDM}{Orthogonal Frequency Division Multiplexing}
\newacronym{cofdm}{COFDM}{Coded Orthogonal Frequency Division Multiplexing}
\newacronym{dvbt2}{DVB–T2}{Digital Video Broadcasting — Second Generation Terrestrial}
\newacronym{em}{EM}{electromagnetic}
\newacronym{icasa}{ICASA}{Independent Communications Authority of South Africa}
\newacronym{ioo}{IOO}{Illuminators of Opportunity}
\newacronym{pr}{PR}{Passive Radar}
\newacronym{qpsk}{QPSK}{Differential Quadrature Phase-Shift Keying}
\newacronym{dqpsk}{DQPSK}{Differential Quadrature Phase-Shift Keying}
\newacronym{etsi}{ETSI}{European Telecommunications Standards Institute}
\newacronym{psk}{PSK}{Phase Shift Keying}
\newacronym{ask}{ASK}{Amplitude-Shift Keying}
\newacronym{fsk}{FSK}{Frequency-Shift Keying}
\newacronym{iq}{IQ}{In-phase and Quadrature}
\newacronym{prs}{PRS}{Phase Reference Symbol}
\newacronym{dft}{DFT}{Discrete Fourier Transform}
\newacronym{fft}{FFT}{Fast Fourier Transform}
\begin{document}
% ----------------------------------------------------
\chapter{Conclusions \label{ch:conclusions}}
\epigraph{The same rule holds for us now, of course: we choose our next world through what we learn in this one. Learn nothing, and the next world is the same as this one.}%
    {\emph{---Richard Bach, Jonathan Livingston Seagull}}
% ----------------------------------------------------

The purpose of this project was to design and implement a \gls{dab} processing chain for the context of a \gls{pr} system. The use of digital broadcasting signals, such as \gls{dab}, in \gls{pr} has become an increasingly attractive topic for researchers. \gls{pr} itself offers a host of advantages over its active radar counterpart, and \emph{digital} broadcasting-based \gls{pr} promises to mitigate some of the problems encountered with \emph{analogue} broadcasting-based \gls{pr}. The exciting field continues to grow.

This report began with a general surveillance of the relevant literature in Chapter~\ref{ch:literature}, in order to paint a backdrop of the project for the reader. This perusal was not about finding assistance to design the \gls{dab} processing chain itself; instead, it was about understanding the \emph{context} for which the chain was being designed. Accordingly, \gls{pr} and digital broadcasting were covered as independent topics, followed by a consideration of the use of the former in the latter: digital broadcasting-based \gls{pr}. It was found that the literature lacked a clear and coherent exposition of the use of \gls{dab} signals in \gls{pr} contexts---focusing too much on either the intricacies of the \gls{dab} standard, or the intricacies of the \gls{pr} implementation. Therefore, the potential value of this project was noted.

The literature review was followed by a richer unpacking of the \gls{dab} standard itself in Chapter~\ref{ch:dab-standard}, beginning with the two main concepts involved in \gls{dab}: \gls{cofdm} and \gls{dqpsk}. Building upon these foundations, a short description of the \gls{dab} transmission frame's structure was then provided. Because of the larger \gls{pr} context in which this project fits, only the relevant aspects of the frame were covered. This minimal approach led to many useful simplifications. The explanations used in this chapter attempted to be as graphical as possible, omitting the often-obfuscating mathematical equations underlying the \gls{dab} signals.

The bulk of the work for this project followed next, in Chapter~\ref{ch:dab-processing}. Here, the \gls{dab} processing chain that was designed in \textsc{Matlab} was documented. Three main blocks of the chain were identified: pre-processing, demodulation, and remodulation. Each of these was constructed from a variety of sub-blocks, which mapped directly to underlying \textsc{Matlab} functions. The pre-processing block was designed to extract a single \gls{dab} frame from a binary file of a \gls{dab} recording; then, with that single frame extracted, the demodulation and remodulation blocks were designed to work in tandem to reconstruct the frame perfectly, according to the \gls{dab} specifications. Of the three blocks, the pre-processing block would likely be the most different in a real-world \gls{pr} system, and this was noted in the chapter. Nonetheless, for all three of the blocks, a thorough account was provided, with detailed block diagrams and sample plots shown. In some places, relevant algorithms were given and discussed; in others, short code-segments were listed. Ultimately, this chapter was intended to act much like a handbook for the \textsc{Matlab} code that was written for the project.

In Chapter~\ref{ch:dab-validation}, the designed \gls{dab} processing chain was then validated. Four different validation approaches were proposed, in increasing rigour. For each of these, an explanation of the philosophy and methodology behind the approach was outlined, followed by its results. Starting at the sub-block level, and moving to the full-chain considerations, all of these validation tests passed. This included the tests done with known reference data. These results thus provided cogent evidence that the \gls{dab} processing chain was indeed designed successfully.

Finally, Chapter~\ref{ch:pr-integration} attempted to provide a short illustrative example of how the \gls{dab} processing chain would fit into a larger \gls{pr} system. A block diagram for this integration was shown, followed by a series of quick simulations and associated plots. This segment was included solely to provide the reader with a bigger-picture understanding of the project and its motivations. No performance analysis was provided.

In summary, the project achieved the goals that were set out, by designing and demonstrating a valid \gls{dab} processing chain. All of the required aspects of the chain were implemented, and omitted functionality was clearly indicated as such.  Moving forwards, this report can indeed act as a guide for the use of \gls{dab} signals in \gls{pr} contexts. Therefore, the project is considered to be a success.

% ----------------------------------------------------
\ifstandalone
\bibliography{../Bibliography/References.bib}
\printnoidxglossary[type=\acronymtype,nonumberlist]
\fi
\end{document}
% ----------------------------------------------------
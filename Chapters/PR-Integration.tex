% ----------------------------------------------------
% PR Integration
% ----------------------------------------------------
\documentclass[class=report,11pt,crop=false]{standalone}
\input{../Style/ChapterStyle.tex}
\makenoidxglossaries

\newacronym{radar}{RADAR}{Radio Detection and Ranging}
\newacronym{dab}{DAB}{Digital Audio Broadcasting}
\newacronym{fm}{FM}{Frequency Modulation}
\newacronym{am}{AM}{Amplitude Modulation}
\newacronym{fdm}{FDM}{Frequency Division Multiplexing}
\newacronym{ofdm}{OFDM}{Orthogonal Frequency Division Multiplexing}
\newacronym{cofdm}{COFDM}{Coded Orthogonal Frequency Division Multiplexing}
\newacronym{dvbt2}{DVB–T2}{Digital Video Broadcasting — Second Generation Terrestrial}
\newacronym{em}{EM}{electromagnetic}
\newacronym{icasa}{ICASA}{Independent Communications Authority of South Africa}
\newacronym{ioo}{IOO}{Illuminators of Opportunity}
\newacronym{pr}{PR}{Passive Radar}
\newacronym{qpsk}{QPSK}{Quadrature Phase-Shift Keying}
\newacronym{dqpsk}{DQPSK}{Differential~Quadrature~Phase-Shift~Keying}
\newacronym{etsi}{ETSI}{European Telecommunications Standards Institute}
\newacronym{psk}{PSK}{Phase Shift Keying}
\newacronym{ask}{ASK}{Amplitude-Shift Keying}
\newacronym{fsk}{FSK}{Frequency-Shift Keying}
\newacronym{iq}{IQ}{In-phase and Quadrature}
\newacronym{ns}{NS}{Null Symbol}
\newacronym{prs}{PRS}{Phase Reference Symbol}
\newacronym{fic}{FIC}{Fast Information Channel}
\newacronym{msc}{MSC}{Main Service Channel}
\newacronym{dft}{DFT}{Discrete Fourier Transform}
\newacronym{idft}{IDFT}{Inverse Discrete Fourier Transform}
\newacronym{fft}{FFT}{Fast Fourier Transform}
\newacronym{ifft}{IFFT}{Inverse Fast Fourier Transform}
\newacronym{fec}{FEC}{Forward Error Correction}
\newacronym{ard}{ARD}{Amplitude-Range-Doppler}
\newacronym{snr}{SNR}{Signal-to-Noise Ratio}
\newacronym{isi}{ISI}{Intersymbol Interference}
\newacronym{mcm}{MCM}{Multicarrier Modulation}
\begin{document}
% ----------------------------------------------------
\chapter{Passive Radar Integration}
% \epigraph{If you wish to make an apple pie from scratch, you must first invent the universe.}%
%     {\emph{Carl Sagan}}
% \epigraph{Philosophers have hitherto only interpreted the world in various ways; the point is to change it.}%
%     {\emph{---Karl Marx}}
\epigraph{Oh, to see without my eyes.}%
    {\emph{---Sufjan Stevens}}
% ----------------------------------------------------

\section{Overview}
This brief chapter serves to provide a rudimentary insight into the context for which the \gls{dab} processing chain was designed: integration into a \gls{pr} system. The scope of this project did not facilitate a thorough analysis into such integration, and the details surrounding its performance, accuracy, efficiency, etc., were not considered. Instead, the work done for this chapter was purely illustrative, included to demonstrate the bigger picture to a reader, in a way that omits dense mathematical analysis. 

\section{System Description}
Recall that the demodulation-remodulation pipeline was not intended for the \emph{decoding} of a \gls{dab} signal---it did not enable one to listen to the audio data of a particular \gls{dab} radio station, for example. Instead, it was to be used in a \gls{pr} system. A conventional, bistatic \gls{pr} system requires two signals for its operation: a reference signal, and a surveillance signal. The former is the signal that arrives via the direct-path, from the uncooperative transmitter to system's receiver. The latter is the signal that is aimed at the scene, as a recording of the scatterings and reflections of the \gls{em} energy emitted by the transmitter. However, when a digital signal is used---one which has an open and accessible standard---the reference data can be generated using the surveillance data. This enables the system to use only one antenna for operations. Various authors have discussed these advantages in great detail~\cite{Fang2018,Barott2014}.

With this in mind, a block diagram depicting the larger \gls{pr}~system's set-up is provided in Figure~\ref{fig:BD_pr-integration}.

\begin{figure}[htbp]
    \centering
    \captionsetup{type=figure}
    \def\svgwidth{\linewidth}
    {\setstretch{0.7} % Line spacing
    \scriptsize
    \input{../Images/BD_pr-integration.pdf_tex}}
    \caption{Block diagram showing }
    \label{fig:BD_pr-integration}
\end{figure}


\section{Simulated Results}

\section{Summary}

% ----------------------------------------------------
\ifstandalone
\bibliography{../Bibliography/References.bib}
\printnoidxglossary[type=\acronymtype,nonumberlist]
\fi
\end{document}
% ----------------------------------------------------